{
\section{Definition of the Brownian web}
\label{sec:brownian-web-definition}

The Brownian web is the continuum scaling limit of a system of
independent-coalescing random walks (see
\cite{tsirelson-lecture-course}).  Constructing the
continuum version raises several technical difficulties addressed in
\cite{fontes-et-al},\cite{norris-turner},\cite{toth-werner}.
Different authors have introduced different state spaces in order to
the give the Brownian web a unique distribution on each of those
spaces (see Theorem 2.1 of \cite{fontes-et-al}, and Theorem 5.1 of
\cite{norris-turner}).
Nonetheless, all the
definitions share the following defining property.

\begin{definition}
\label{def:web}
\newcommand{\simplex}{\mathcal{S}}

  Denote $\simplex=\{(s, t) \in \R^2 : s \le t\}$.
\begin{itemize}
  \item
  A Brownian web on a probability space $\Omega$ induces a (jointly)
  measurable mapping $\webnoargs : \Omega \cross \simplex \cross \R
  \to \R$, $(\omega, (s, t), x) \mapsto \web{s}{t}{x}$ (suppressing
  $\omega$ in the notation).

  \item
  On every finite collection of
  starting points $(s_1, x_1),(s_2, x_2),...,(s_n, x_n)$, the
  collection of processes $\web{s_1} {\cdot}{x_1},
  \web{s_2}{\cdot}{x_2},...,\web{s_n}{\cdot}{x_n}$
  forms a system of $n$ independent-coalescing Brownian motions.

  \item
  There exists a countable subset $S \subseteq \R^2$ such that the
  collection $\{\web{s}{t}{x} : s,t \in S\}$ generates the entire
  \sigfield{} of the Brownian web.
\end{itemize}

  \newcommand{\bm}[1]{X^{#1}}
  A system of $n$ independent-coalescing Brownian motions is a finite
  collection of stochastic processes $(\bm{1}, \bm{2},...,\bm{n})$ such that
  each $\bm{i}$ starts at some point $x_i$ at some time $s_i$, and $(\bm{1},
  \bm{2},...,\bm{n})$ are independent until the first time $T$ at which
  $\bm{i}(T)=\bm{j}(T)$ for some $i\neq j$. From this time onwards $\bm{i}(T)$
  and $\bm{j}(T)$ coalesce and continue with the rest of the $\bm{k}$ (for
  $k\neq i,j$) as a system of $n-1$ independent-coalescing Brownian motions.
\end{definition}

To observe that the last property is indeed a property of the Brownian
web, see for example equation (2.6) in \cite{toth-werner}.  In this
equation we can see that $\Lambda$ (which is the same as our
$\webnoargs$) can be recovered from the countable collection $F_n =
\Lambda_{(s_n,t_n)}$.  There $\{(s_n,t_n) : n \in \mathbb{N}\}$ is
some ordering of $D \times D$ where $D \subseteq \R^2$ is the
collection of diadic rationals.

For the equivalent construction in \cite{fontes-et-al}, item (2) of
Theorem~2.1 guarantees that the third property in our definition
holds.

  Several trajectories of a Brownian web can be seen in Figure
  \ref{fig:bw-trajectories}.

\begin{figure}
   \centering
   \includegraphics[scale=2]{sometraj.eps}
   \caption{Some trajectories of the Brownian web. A particular trajectory is marked.}
  \label{fig:bw-trajectories}
\end{figure}
}
