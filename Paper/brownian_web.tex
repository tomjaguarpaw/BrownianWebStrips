{
\newcommand{\factor}[2]{\mathcal{F}_{#1 #2}}
\newcommand{\commafactor}[2]{\mathcal{F}_{#1,#2}}
\newcommand{\simplex}{\mathcal{S}}

\section{Definition of the Brownian web on horizontal strips}

\begin{definition}
  A pair of independent-coalescing Brownian motions is a pair of
  stochastic processes $(X, X')$ such that $X_t$ is defined for $t \in
  [s, \infty)$, say, and $X'_t$ for $t \in [s', \infty)$, say.  $X$
      and $X'$ are Brownian motions independent until time $T = \inf
      \{ t \ge s \max s' : X_t = X'_t\}$, and for $t \ge T$ $X_t =
      X'_t$.
\end{definition}

\begin{definition}
  Let $\simplex$ = $\{(t, s) \in \R^2 : t \ge s\}$
\end{definition}

\begin{definition}
  A Brownian web on a probability space $\Omega$ is a
  jointly-measurable mapping $\phi : \Omega \cross \simplex \cross
  \R \to \R$, $(\omega, t, s, x) \mapsto \phi_{ts}(x)$ (we supress
  $\omega$ in the notation) such that for fixed pairs of starting
  points $(s, x)$ and $(s', x')$, the pair of processes $(\phi_{\cdot
    s}(x), \phi_{\cdot s'}(x'))$ is a pair of independent-coalescing
  Brownian motions.
\end{definition}

\begin{definition}
  \label{def:restrict}
  From a path $f : [s, \infty) \to \R$ we define ``$f$ stopped on
    leaving the interval $I$'' to be the map $\restrict{I}{f} : [s,
      \infty) \to \R$ given by $\restrict{I}{f}(t) = f(t \minsym T)$
      where $T = \inf\{ t : f(t) \not\in I \}$.
\end{definition}

\newcommand{\brownianwebnoise}{collection of horiztontal
  sub-$\sigma$-algebras of the Brownian web}

\begin{definition}
  The \brownianwebnoise{} is an
  association of a $\sigma$-algebra $\factor{z}{y}$ to each $(z, y)
  \in \simplex$, generated by the following collection of paths:
  $\{ \restrict{(y,z)}{t \mapsto \phi_{ts}(x)} : s, x \in \R \}$
\end{definition}

\begin{theorem}
  The \brownianwebnoise{}
  is a continuous product of probability spaces in the sense that
  for each $x \le y \le z$, \FIXME{$\factor{z}{y} \tensor \factor{y}{x} =
  \factor{z}{x}$}{Define this} \FIXME{}{Up to sets of measure $0$?}.
  See 3c1 Definition of
  \cite{tsirelson-nonclassical-stochastic-flows}.
\end{theorem}

\begin{theorem}
  The \brownianwebnoise{} is a noise in the sense that it is a
  continuous product of probability spaces with a translation.
  See 3d1 Definition of
  \cite{tsirelson-nonclassical-stochastic-flows}.
\end{theorem}

\begin{theorem}
  The collection of two-dimensional sub-$\sigma$-algebras of the
  Brownian web is a two-dimensional noise, and moreover it is black.
\end{theorem}

\begin{proof}
  We know that the one-dimensional noise consisting of the vertical
  collection of sub-$\sigma$-algebras is black.  By general arguments
  on spectral sets this implies that the two-dimensional noise is
  black.  See \cite{tsirelson-classicality-blackness-spectrum}.
\end{proof}

\section{Definition of the ``resampled'' process}

\newcommand{\reservoir}{\mathcal{G}}

\newcommand{\twostrips}{\commafactor{\infty}{0} \tensor \commafactor{0}{-\infty}}
\newcommand{\onestrip}{\commafactor{\infty}{-\infty}}
\newcommand{\twostripsreservoir}{\twostrips \tensor \reservoir}

\begin{definition}
Given a Brownian web $\phi$, fix a starting point $(s,x)$ and write
$\sampled$ for the random process $t \mapsto \phi_{ts}(x)$\FIXME{}{say
  ``which is a Brownian motion''?}.
\end{definition}

{
\newcommand{\webt}{\psi}
\begin{definition}
  We define $\resamplede$, a ``perturbed'' version of $\sampled$, as
  follows.

  Let $\webt$ be a Brownian web independent of $\phi$ (measurable with
  respect to $\reservoir$, say, where $\reservoir$ is independent of
  $\commafactor{\infty}{-\infty}$).

  \begin{itemize}
  \item Starting from time $s$, follow the path $\phi_{\cdot s}(x)$
    until it hits $0$, at time $S_1$, say.
  \item Then follow $\webt_{\cdot S_1}(0)$ until it hits $\pm \epsilon$, at
    time $T_1$, say.
  \item Then follow $\phi_{\cdot T_1}(\pm \epsilon)$ until it hits $0$, at
    time $S_2$, say.
  \item Continue indefinitely in this inductive fashion.
  \end{itemize}
\end{definition}

\TODO{Note that $\webt$ need not be a Brownian web.  All the we need
  is that the trajectories are $\reservoir$-measurable Brownian
  motions.}{}
}

\begin{obs}
  \label{obs:2d-proc}
  A rough description of the coupled pair $(\sampled, \resamplede)$
  follows.

  \FIXME{}{}
\end{obs}

\begin{obs}
  $\resamplede$ is $\twostripsreservoir$-measurable.
\end{obs}

\begin{lemma}
  $\resamplede \to \sampled$ uniformly on compacts in probability as $\epsilon \to 0$.
\end{lemma}

\begin{cor}
  \label{cor:sampled-twostripsreservoir-meas}
  $\sampled$ is $\twostripsreservoir$-measurable.
\end{cor}

\begin{theorem}
  $\twostrips = \onestrip$
\end{theorem}

\begin{proof}
  $\onestrip$ is generated by random variables of the form $X$, thus
  by Corollary \ref{cor:sampled-twostripsreservoir-meas} $\onestrip
  \subseteq \twostripsreservoir$.  Since $\onestrip$ is
  independent of $\reservoir$ a simple result on tensor products of
  Hilbert spaces (for example (1c1), p5 of
  \cite{tsirelson-completion}) shows
  that $\onestrip \subseteq \twostrips$.
\end{proof}
}
