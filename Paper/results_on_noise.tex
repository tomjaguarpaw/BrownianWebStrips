{
\section{Conclusions about the noise}
\label{sec:conclusions-about-the-noise}

We conclude by supplying a formal framework for the statement of
Theorem \ref{thm:bw-2d-black-noise} followed by its proof.
The following definition of noise is a straightforward extension of
that due to Tsirelson (Definition
3d1 of \cite{tsirelson-nonclassical-stochastic-flows})
to multiple dimensions.

\newcommand{\F}{\mathcal{F}}
A \emph{$d$-dimensional noise} consists of a probability space $(\Omega,\F,
\P)$, sub-$\sigma$-fields $\F_R \subset \F$ given for all (open)
$d$-dimensional rectangles $R \subset \R^d$, and a measurable action
$(T_h)_h$ of the additive group of $\R^d$ on $\Omega$, having the following properties:

\begin{enumerate}[(a)]
\item \label{item:tensor-condition} $\F_R \tensor \F_{R'} = \F_{R''}$ whenever
$R$ and $R'$ partition $R''$, in the sense that
$R\cap R'=\emptyset$ and the closure of $R \cup R'$
is the closure of $R''$,
\item \label{item:translation-condition} $T_h$ sends $\F_R$ to $\F_{R+h}$ for each $h \in \R^d$,
\item \label{item:generation-condition} $\F$ is generated by the union of all $\F_R$.
\end{enumerate}

When $d = 1$ our definition coincides with that of Tsirelson.
In that case, $R$ ranges over all open intervals
and condition~(\ref{item:tensor-condition}) translates to
$\F_{(s,t)} \tensor \F_{(t,u)} = \F_{(s,u)}$ whenever $s < t < u$.

As conditions (\ref{item:translation-condition}) and
(\ref{item:generation-condition}) are immediate for
the horizontal factorization of the Brownian web,
Theorem~\ref{thm:recoveringfromstrips} immediately
implies the following:

\begin{proposition*}
The horizontal factorization of the Brownian web is a (1-dimensional) noise.
\end{proposition*}

Recall that the horizontal factorization of the Brownian web is an
association of a \sigfield{} to any horizontal strip (see
Definition~\ref{def:horizontal-factorization}).
Observe that the association arises from considering trajectories of the Brownian
web stopped at the first time they are outside a particular strip.
Similarly, we can associate a \sigfield{} to any vertical strip, or
indeed to any rectangle.  The former association is the \emph{vertical
  factorization of the Brownian web} and the latter the
\emph{two-dimensional factorization}.

Recall that the horizontal factorization of the Brownian web is an
association of a \sigfield{} to any horizontal strip (see
Definition~\ref{def:horizontal-factorization}).
Observe that the association arises from considering trajectories of the Brownian
web stopped at the first time they are outside a particular strip.
Similarly, we can
associate a \sigfield{} to any vertical strip, or indeed to any rectangle.
The former association is the \emph{vertical factorization of the Brownian web}
and the latter the \emph{two-dimensional factorization}.

We can extend existing results to derive the following:

\begin{proposition*}
The Brownian web factorized on two-dimensional
rectangles is a two-dimensional noise.
\end{proposition*}

That is, when a rectangle is partitioned horizontally or vertically
into two smaller rectangles, the \sigfield{} of the larger is
generated by the \sigfield{}s of the two smaller.  To see this holds
for a rectangle partitioned by a horizontal split observe that this is
a consequence of our result restricted to a finite time interval.  The
see it holds  for a vertical split, observe that this is a
straightforward modification of the earlier result that the vertical
factorization of the Brownian web is a noise (see, for example,
\cite{tsirelson-scaling-limit-noise-stability}).

Furthermore, by a general result
of Tsirelson (see
\cite{tsirelson-noise-as-a-boolean-algebra} Theorem 1e2),
a two-dimensional noise is black when
one of its one-dimensional factorizations is
black. As the vertical factorization of
the Brownian web is black (see
\cite{tsirelson-nonclassical-stochastic-flows}),
we deduce Theorem \ref{thm:bw-2d-black-noise}.
}
