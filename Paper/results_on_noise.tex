{
\section{Conclusions about the noise}
\label{sec:conclusions-about-the-noise}

We conclude by supplying a formal framework for the statement of
Theorem \ref{thm:bw-2d-black-noise} followed by its proof.

\newcommand{\F}{\mathcal{F}}
A $d$-dimensional noise consists of a probability space $(\Omega,F, \P)$, sub-$\sigma$-fields $\F_R \subset \F$ given for
all (open) $d$-dimensional rectangles $R \subset \R^d$, and a measurable action $(T_h)_h$ of $\R^d$ on, having the following
properties:
\begin{enumerate}
\item \label{item:tensor-condition} $\F_R \tensor \F_{R'} = \F_{R''}$ whenever $R''$ is a
rectangle partitioned by rectangles $R$ and $R'$ in the sense that
$R\cap R'=\emptyset$ and the closure of $R \cup R'$
is the closure of $R''$,
\item \label{item:translation-condition} $T_h$ sends $\F_R$ to $\F_{R+h}$ for each $h \in \R^d$,
\item \label{item:generation-condition} $\F$ is generated by the union of all $\F_R$.
\end{enumerate}

In the case of a one-dimensional noise, $R$ ranges over all open intervals
and condition~(\ref{item:tensor-condition}) translates to
$\F_{(s,t)} \tensor \F_{(t,u)} = \F_{(s,u)}$ whenever $s < t < u$.

As conditions (\ref{item:translation-condition}) and
(\ref{item:generation-condition}) are immediate for
horizontal factorization of the Brownian web (see
Definition~\ref{def:horizontal-factorization}),
Theorem~\ref{thm:recoveringfromstrips} immediately
implies the following:

\begin{theorem}
The Brownian web factorized on horizontal strips is a noise.
\end{theorem}

We can also extend the definition of the horizontal
factorization of the web, defining a two-dimensional
factorization into rectangles along the lines of the
same definition. As the proof of Theorem \ref{thm:recoveringfromstrips}
holds when restricted to vertical strips, and the proof of its vertical
counterpart holds when restricted to horizontal strips (see \TODO{}{add
ref if possible}), we can extend our results to derive the following:

\[\text{The Brownian web, factorized on two-dimensional
rectangles is a two-dimensional noise.}\]
\FIXME{}{Break the line}

Furthermore, by a completely general and abstract result
of Tsirelson \TODO{}{cite}, a two-dimensional noise which is
black in one of its one-dimensional factorizations is also
black. As this holds for the Brownian web (see \TODO{cite}),
we deduce Theorem \ref{thm:bw-2d-black-noise}.

\TODO{\section{Remarks and open problems}}{This belongs in a different file, perhaps}

\label{sec:open-problems}

In this paper we present the second known example of a two-dimesional
black noise, after the scaling limit of critical planar percolation,
as proved by Schramm and Smirnov.  In their result they also proved
that \sigfield{}s can be associated to a larger class of domains than
just rectangles in a way that still allows the \sigfield{} of a larger
domain to be recovered from the \sigfield{}s of two smaller domains
that partition it, as long as those domains all have boundaries of
Hausdorff dimension strictly less than \FIXME{5/4}{check remark 1.8 in
  Schramm-Smirnov}.  This raises the following question:

\begin{openproblem}
  To what class of domains can the Brownian web noise be extended?
\end{openproblem}

We suspect the answer to be more sophisticated than for percolation,
since the Brownian web is not rotationally invariant.  
\TODO{}{Check what is known about the web and Schramm-Smirnov}
\TODO{}{Write about different stability on different axes}

Additionally we would like to ask,

\begin{openproblem}
  Can we demonstrate an explicit example of domains to which the noise
  cannot be extended?
\end{openproblem}

By general results, such domains must exist, \FIXME{see Tsirelson}{}.

Furthermore, this discovery calls for seeking more examples of
two-dimensional black noises, and indeed in higher dimensions.  Their
discovery would shed light on the nature of black noises.

\begin{openproblem}
  Find more examples of two-dimensional black noises.  Show an example
  of a black noise in three dimensions or more.
\end{openproblem}

\TODO{\section{Acknowledgements}}{This belongs in a different file,
  perhaps}

\label{sec:acknowledgements}

The authors would like to thank Boris Tsirelson for his guidance in
the field of noises, introducing the question to us, and helpful
discussions.  In addition we would like to thank Ron Peled and Nomi
Feldheim for simplifying some of our arguments and their useful
comments on preliminary drafts.
}
