{
\section{Recovering the Brownian web from its strips}

\begin{definition}
  \newcommand{\T}{\stripleavetime{f}}
  \label{def:restrict}
  From a path $f : [s, \infty) \to \R$ we define ``$f$ stopped on
    leaving the interval $(a,b)$'' to be the map $\restrict{a}{b}{f} : [s,
      \infty) \to \R$ given by $\restrict{a}{b}{f}(t) = f(t \minsym \T)$
      where $\T = \inf\{ t : f(t) \not\in (a,b) \}$.
\end{definition}

\newcommand{\brownianwebnoise}{collection of horiztontal
  sub-$\sigma$-algebras of the Brownian web}

\begin{definition}
  The \brownianwebnoise{} is an
  association of a $\sigma$-algebra $\factor{y}{z}$ to each $(y, z)
  \in \simplex$, generated by the following collection of paths:
  $\{ \restrict{y}{z}{t \mapsto \web{s}{t}{x}} : s, x \in \R \}$
\end{definition}

For notational convenience in the rest of this
section we will fix $a \le b \le c$ and write $\Res{\cdot}$ for
$\restrict{a}{c}{\cdot}$.

\begin{observation}
  $\Res{\resamplede}$ could be called the ``$\epsilon$-resampled
  truncated process'' and is $\factor{a}{b} \tensor \factor{b}{c}
  \tensor \reservoir$-measurable.
\end{observation}

\begin{lemma}
  $\Res{\resamplede} \to^\P \Res{\sampled}$ as $\epsilon \to 0$
\end{lemma}

\begin{proof}
  We could show this directly along the lines above.  However instead
  we will do it as follows.  Note that
  \begin{itemize}
  \item $\resamplede \to^\P \sampled$
  \item the probability that $f \mapsto \Res{f}$ is continuous
    at $X$ is $1$
  \end{itemize}
  so we conclude that $\Res{\resamplede} \to^\P
  \Res{\sampled}$ by the continuous mapping theorem (see for
  example \cite{billingsley}, p. 21, Theorem 2.7).
\end{proof}

\begin{lemma}
  $\Res{\sampled}$ is $\factor{a}{b} \tensor
  \factor{b}{c}$-measurable.
\end{lemma}

\begin{lemma}
  If $\stripleavetime{f}$ is not a turning point of the path $f$,
  then the map $f \mapsto \Res{f}$ is continuous at $f$ in the
  topology of uniform convergence on compacts.
\end{lemma}

\begin{proof}
  \DOTHIS{We sketch a proof of this straightforward result in
    classical analysis.

    \newcommand{\fn}{f_n}

    \newcommand{\T}{\stripleavetime{f}}
    \newcommand{\Tn}{\stripleavetime{\fn}}

    Suppose $\fn \to f$ uniformly on compacts.  Then since $\T$ is not
    a turning point of $f$ it is easily seen that $\Tn \to \T$ and
    that eventually $\fn(\Tn) = f(\T)$.

    \renewcommand{\d}{\delta}
    \newcommand{\e}{\epsilon}
    \newcommand{\starttime}{-}

    Since $f$ is continuous, choose $\d$ such that
    $|f(\T) - f(x)| \le \epsilon$ when
    $|\T - x| \le \delta$.  Eventually
    \begin{itemize}
    \item $\fn(\Tn) = f(\T)$
    \item $|\Tn - \T| \le \d$
    \item $|\fn - f| \le \e$ uniformly on some compact containing
      $[\starttime, \T+\d]$.
    \end{itemize}

    \newcommand{\stoppedfn}{\Res{\fn}}
    \newcommand{\stoppedf}{\Res{f}}

    \newcommand{\condition}[2]{for $t \in {#1}$ we have
      $|\stoppedfn(t) - \stoppedf(t)| #2$}
    so
    \begin{itemize}
    \item \condition{[\starttime, \T-\d)}{\le \e}
    \item \condition{[\T-\d, \T+\d]}{\le 2\e}
    \item \condition{(\T+\d, \infty)}{= 0}
    \end{itemize}
    so indeed $\stoppedfn \to \stoppedf$ uniformly on compacts (indeed
    globally since they are both eventually $0$ outside some
    compact).}{The only tricky part is the second item.  There are
      four cases to consider.  The worst case is when $\stoppedf$ has
      been truncated and $\stoppedfn$ hasn't.  But $\stoppedf$ is
      equal to the level, and so is within $\e$ of $f$.  OTOH
      $\stoppedfn = \fn$ so is within $\e$ of $f$.  Thus the $2\e$
      bound.}
\end{proof}
\subsection{What does this mean in terms of noises?}

\begin{theorem}
  $\factor{a}{c} = \factor{a}{b} \tensor \factor{b}{c}$
\end{theorem}

\begin{theorem}
  The \brownianwebnoise{}
  is a continuous product of probability spaces in the sense that
  for each $x \le y \le z$, $\factor{x}{y} \tensor \factor{y}{z} =
  \factor{x}{z}$.
  See 3c1 Definition of
  \cite{tsirelson-nonclassical-stochastic-flows}.
\end{theorem}

\begin{theorem}
  The \brownianwebnoise{} is a noise in the sense that it is a
  continuous product of probability spaces with a translation.
  See 3d1 Definition of
  \cite{tsirelson-nonclassical-stochastic-flows}.
\end{theorem}

\FIXME{}{Need to make a comment that the obvious translation works.}

\begin{theorem}
  The collection of two-dimensional sub-$\sigma$-algebras of the
  Brownian web is a two-dimensional noise, and moreover it is black.
\end{theorem}

\begin{proof}
  We know that the one-dimensional noise consisting of the vertical
  collection of sub-$\sigma$-algebras is black.  By general arguments
  on spectral sets this implies that the two-dimensional noise is
  black.  See \cite{tsirelson-classicality-blackness-spectrum}.
\end{proof}

}
