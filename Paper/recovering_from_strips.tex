{
\section{Recovering the Brownian web from its strips}

\begin{definition}
  \newcommand{\T}{\stripleavetime{f}}
  \label{def:restrict}
  From a path $f : [s, \infty) \to \R$ we define ``$f$ stopped on
    leaving the interval $(a,b)$'' to be the map $\restrict{a}{b}{f} : [s,
      \infty) \to \R$ given by $\restrict{a}{b}{f}(t) = f(t \minsym \T)$
      where $\T = \inf\{ t : f(t) \not\in (a,b) \}$.
\end{definition}

\newcommand{\brownianwebnoise}{collection of horiztontal
  sub-$\sigma$-algebras of the Brownian web}

\begin{definition}
  The \brownianwebnoise{} is an
  association of a $\sigma$-algebra $\factor{y}{z}$ to each $(y, z)
  \in \simplex$, generated by the following collection of paths:
  $\{ \restrict{y}{z}{t \mapsto \web{s}{t}{x}} : s, x \in \R \}$
\end{definition}

\begin{theorem}
  The \brownianwebnoise{}
  is a continuous product of probability spaces in the sense that
  for each $x \le y \le z$, $\factor{x}{y} \tensor \factor{y}{z} =
  \factor{x}{z}$.
  See 3c1 Definition of
  \cite{tsirelson-nonclassical-stochastic-flows}.
\end{theorem}

\begin{theorem}
  The \brownianwebnoise{} is a noise in the sense that it is a
  continuous product of probability spaces with a translation.
  See 3d1 Definition of
  \cite{tsirelson-nonclassical-stochastic-flows}.
\end{theorem}

\FIXME{}{Need to make a comment that the obvious translation works.}

\begin{theorem}
  The collection of two-dimensional sub-$\sigma$-algebras of the
  Brownian web is a two-dimensional noise, and moreover it is black.
\end{theorem}

\begin{proof}
  We know that the one-dimensional noise consisting of the vertical
  collection of sub-$\sigma$-algebras is black.  By general arguments
  on spectral sets this implies that the two-dimensional noise is
  black.  See \cite{tsirelson-classicality-blackness-spectrum}.
\end{proof}

}
