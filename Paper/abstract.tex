\newcommand{\sigfield}{$\sigma$-field}
{
\title{The Brownian web is a two-dimensional black noise}

\newcommand{\tomthanks}{This work was supported in part by
a Wyndham Deedes Memorial Travel Scholarship from The Anglo-Israel
Association.}

\newcommand{\ohadthanks}{School of Mathematics, Raymond and Beverly Sackler Faculty of Exact
Sciences, Tel Aviv University, Tel Aviv, Israel. E-mail:
ohad\_f@netvision.net.il. Research supported by an ERC advanced grant.}

\author{Tom Ellis\thanks{\tomthanks}\\%
\and Ohad N. Feldheim\thanks{\ohadthanks}}

\date{2011}

\maketitle

\begin{abstract}
We answer a question of
Tsirelson \cite{tsirelson-nonclassical-stochastic-flows} Section 11b.

\RON{}{Elaborate}
\end{abstract}

\section{Introduction}
In this paper we study a stochastic object called the \emph{Brownian web}. We
research this object in the context of the theory of classical and
non-classical noises, developed by Boris Tsirelson [] [] []. Our main result
is, that in the terminology of this theory, the Brownian web is a
two-dimensional \emph{black} noise.
Roughly speaking, the Brownian web is a random variable assigning to
every space-time point in $\R\times\R$, a standard Brownian motion starting
at that point. Each of those motions are independent until one hits another,
and after that point those two coalesce, continuing together. This object was
originally studied more then twenty-five years ago by Arratia [], motivated
by a study of the asymptotics of one-dimensional voter models, and later on
by T\'{o}th and Werner, motivated by the problem of constructing continuum
``self-repelling motions'', and by Fontes, Isopi, Newman and Ravishankar,
motivated by its relevance to ``aging'' in statistical physics of
one-dimensional coarsening, \FIXME{and by Norris and Turner}{}.
A rigorous notion of Brownian webs in our context
can be found in \TODO{}{give reference}.

The Brownian web functions as an important example in the theory of
classical and non-classical noises developed by Tsirelson []. In this
theory a noise is roughly a probability space along with a collection
of sub-\sigfield{}s indexed by open rectangles (possibly infinite) of
$\R^d$.  The sub-\sigfield{} associated to a rectangle is intended to
represent the behavior of a stochastic object within that rectangle.
The \sigfield{}s must satisfy the following three properties:
the \sigfield{}s associated to disjoint rectangles of $\R^d$ are
independent; translations on $\R^d$ act in a way that preserves the
probability measure on each \sigfield{}; and the \sigfield{}
associated to a rectangle is generated by the two \sigfield{}s
associated to two smaller rectangles which partition it.
Two natural examples of a noises are the Gaussian white noise
and the Poisson noise. Those two noises are called classical, or white,
meaning that
resampling of a small portion of $\R^d$ doesn't change the observables of the
process very much.

The foundational result of Tsirelson and Vershick showed that there
exist non-classical noises.  Indeed there exist non-classical noises
that are as far from white as could be, and these we call black.  The
defining property of a black noise is that all its observables are
unstable, i.e.\ for any particular observable, resampling a small portion of the noise can
render that observable nearly independent of its original
value.  (For a more formal discussion of black and white, classical
and non-classical noises see \TODO{}{}.  As definitions of black noise are
hard to find we also supply a full definition of a black noise in
the \TODO{appendix}{do the appendix}.)  Later the Brownian web,
when considered as a time-indexed random process,
was shown by Tsirelson in \TODO{}{} to be a one-dimensional black noise.  We
extend this
result by showing that the Brownian web is in fact a two-dimensional
black noise.  Formally:

\begin{theorem}
\label{thm:bw-2d-black-noise}
The Brownian web is a
two-dimensional black noise.
\end{theorem}

This makes the Brownian web only the second
known two-dimensional black noise after Schramm and Smirnoff's
recent result on the scaling limit of critical planar
percolation \TODO{}{citation}.

Of the three properties required for a process to be a noise, the
first two, i.e.\ translation invariance and independence on disjoint
domains, hold trivially for the Brownian web.  Furthermore, once we
have shown that the Brownian web is a two-dimensional noise, it will
follow that it is a two-dimensional \emph{black} noise, since a
sufficient condition for a two-dimensional noise to be black is that
one of its one-dimensional projections is black.

The main difficulty in proving
Theorem \ref{thm:bw-2d-black-noise} is to show that the $\sigma$-field
associated to any rectangle is generated by the two $\sigma$-fields
associated to any two rectangles that partition it.
The major milestone towards this result is to show the special case
when the large rectangle is the whole plane, and the smaller
rectangles that partition it are the upper and lower half-planes.

\begin{theorem}
\label{thm:informal-recovering-from-half-planes}
In the Brownian web, the $\sigma$-field associated to the whole plane
is generated by that associated to the upper half-plane and that
associated to the lower half-plane.
\end{theorem}

The formal counterpart of this is
Theorem \ref{thm:recoveringfromhalfplanes} (whose statement also
contains the independence condition).

The rest of the paper goes as follows: In Section \ref{sec:brownian-web-definition} we define the
Brownian web formally; in Section \ref{sec:recovering-from-half-planes} we state Theorem \ref{thm:recoveringfromhalfplanes}, the formal
counterpart of Theorem \ref{thm:informal-recovering-from-half-planes}; we then reduce this theorem to a
convergence result for an auxiliary process which we prove in Section
\ref{sec:proof-of-lem:resamplede-to-sampled}.  In Section \ref{sec:recovering-from-strips} we extend Theorem \ref{thm:informal-recovering-from-half-planes} to hold for the
$\sigma$-fields associated to horizontal strips as well; in Section \ref{sec:conclusions-about-the-noise}
we extend further to all rectangles.  In addition we define noises
properly and conclude by proving Theorem \ref{thm:bw-2d-black-noise}.  Section \ref{sec:open-problems} is devoted to
remarks and open problems and Section \ref{sec:acknowledgements} contains some
acknowledgements.

}
