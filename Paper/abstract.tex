{
\title{The Brownian web is a two-dimensional black noise}

\newcommand{\tomthanks}{This work was supported in part by
a Wyndham Deedes Memorial Travel Scholarship from The Anglo-Israel
Association.}

\newcommand{\ohadthanks}{School of Mathematics, Raymond and Beverly Sackler Faculty of Exact
Sciences, Tel Aviv University, Tel Aviv, Israel. E-mail:
ohad\_f@netvision.net.il. Research supported by an ERC advanced grant.}

\author{Tom Ellis\thanks{\tomthanks}\\%
\and Ohad N. Feldheim\thanks{\ohadthanks}}

\date{2011}

\maketitle

\begin{abstract}
We answer a question of
Tsirelson \cite{tsirelson-nonclassical-stochastic-flows} Section 11b.

\RON{}{Elaborate}
\end{abstract}

\section{Introduction}
In this paper we study a stochastic object called the \emph{Brownian web}. We
research this object in the context of the theory of classical and
non-classical noises, developed by Boris Tsirelson [] [] []. Our main result
is, that in the terminology of this theory, the Brownian web is a
two-dimensional \emph{black} noise.
 Roughly speaking, the Brownian web is a random variable assigning to
every space-time point in $\R\times\R$, a standard Brownian motion starting
at that point. Each of those motions are independent until they hit each
other, and after that they coalesce, continuing together. This object was
originally studied more then twenty five years ago by Arratia [], motivated
by a study of the asymptotics of one dimensional voter models, and later on
by T\'{o}th and Werner, motivated by the problem of constructing continuum
"self-repelling motions", and by Fontes,Isopi, Newman and Ravishankar,
motivated by its relevance to ``aging'' in statistical physics of
one-dimensional coarsening. A rigorous notion of Brownian webs in our context
can be found in \TODO{}{give reference}.

The Brownian web functions as an important example in the theory of classical
and non-classical noises developed by Tsirelson []. In this theory a noise is
roughly a collection of random sigma-field indexed on subsets of $\R^d$,
distribution is independent on disjoint subsets of $\R^d$, translation
invariant on $\R^d$, and satisfies that for subfield of a $d$-dimensional
rectangle, and for every partition of it to two smaller rectangles, the $\sigma$-algebra
of the bigger rectangle is measurable in respect to the fields of the
smaller ones. Two natural examples of a noises, are the Gaussian white noise,
and the Poisson noise. Those two noises are white in the sense that a
resampling of a small portion of $\R^d$ doesn't change the observables of the
process very much. The Brownian web, however, was shown by Tsirelson in [] to
be a one-dimensional black noise, that is, a noise for which a certain
resampling method which changes only an arbitrarily small portion of $\R^d$,
renders every observable of the result independent of its original value. For
a more formal discussion of black and white noises see []. As definitions of
black noise are hard to find we also supply a full definition of a black
noise in the appendix.

Our main result in this work is the following theorem:

\begin{theorem}
The Brownian web, factorized on two-dimensional rectangles is a
two-dimensional black noise.
\end{theorem}

\TODO{\subsection{Factorizing the web}}{}
}
