{
\section{Ron's argument}

\newcommand{\bandwidth}{\delta}
\newcommand{\rotproc}{Y}

\newcommand{\union}{\cup}
\renewcommand{\L}{L^+ \union L^-}
\newcommand{\Le}{L^\epsilon}

This section is devoted to proving Lemma
\ref{lem:resamplede-to-sampled}, i.e.\ that
\statementoflemresampledetosampled.

It will be slightly easier to work with the process $\rotproc =
(\sampled + \resamplede, \sampled - \resamplede)$ which is $(\sampled,
\resamplede)$ rotated clockwise about the origin by $\pi / 4$ radians, and enlarged by a factor of
$\sqrt{2}$.

\newcommand{\boundarylines}{A}

Choose $\delta$ and let $\boundarylines$ be the union of the
horizontal lines at levels $\pm\delta$.  To prove the convergence
result we will first show that by choosing $\epsilon$ small enough we
can make arbitrarily small the probability that $\rotproc$ has hit
$\boundarylines$ before time $1$.

Assume for convenience that the processes start at time $0$.  At time
$0$, $\sampled$ and $\resamplede$ take the same value, so $\rotproc$
starts on the horizontal axis.  It remains on the horizontal axis
performing a Brownian motion of volatility $2$, until it reaches $0$.
At this point it becomes a standard two-dimensional Brownian motion.
It returns to being a one-dimensional Brownian motion on the
horizontal axis when it is ``absorbed''.  To be absorbed means that it
hits $\L$, or it hits $\Le$ when $\resamplede$ has more recently taken
value $\pm\epsilon$ than $0$.  (When this condition holds we will say
that $\Le$ is in an ``absorbing state'').  This corresponds to
$\resamplede$ moving according to the web $\webnoargs$ and coalescing
with $\sampled$.

\begin{definition}
  We say $Y$ is \emph{absorbed} when it hits $\L$, or it hits $\Le$
  when it is in an absorbing state.
\end{definition}

Our argument does not actually depend on $\Le$ ever being in an
absorbing state.  Our intuition tells us that having $\Le$ being
absorbing can only ever help $\rotproc$ remain close to the horizontal
axis.

\newcommand{\farpoint}{(P,0)}
\newcommand{\origin}{(0,0)}

We will consider $Y$ as a process which performs a sequence of
excursions, whereby it starts at $\origin$, is later absorbed, and
then eventually returns to $\origin$ again to start another excursion.

\begin{definition}
  An excursion of $Y$ is the behaviour of $Y$ between two stopping times
  $S \le T$ such that $Y_S =
  \origin$ and $T = \inf\{ t > S : Y_t = \origin \}$.
\end{definition}

Note that there is a.s.\ no excursion for which $S = T$, since almost
surely $Y$ spends a positive amount of time away from $\origin$ before
returning.

\newcommand{\Omegaeloge}{\Omega(\epsilon\log\epsilon)}

The key to the proof is that during an excursion $\rotproc$ hits
$\boundarylines$ with probability $O(\epsilon)$, but hits some fixed
point on the horizontal axis, $\farpoint$ say, with a probability
$\Omegaeloge$ which is eventually much higher.  The following two
lemmas state these results and their proofs are presented at the end
of the section.

\begin{lemma}
  \label{lem:Phitboundaryline}
  The probability that during an excursion $Y$ hits $\boundarylines$
  is $O(\epsilon)$.
\end{lemma}

\begin{lemma}
  \label{lem:Pabsorbedandtravelsfar}
  Fix $P > 0$.  The probability that during an excursion $Y$ hits $\farpoint$
  is $\Omegaeloge$.
\end{lemma}

\begin{lemma}
  Fix $P > 0$.  The probability that during an excursion $Y$ hits $\farpoint$
  and does not hit $A$ is $\Omegaeloge$.
\end{lemma}

\begin{proof}
  By Lemmas \ref{lem:Phitboundaryline} and
  \ref{lem:Pabsorbedandtravelsfar} the probability is at least
  $\Omegaeloge - O(\epsilon) = \Omegaeloge$.
\end{proof}

Now instead of thinking of excursions separately, we show that with
high probability $\rotproc$ hits $\farpoint$ before $\boundarylines$.

\begin{lemma}
  Fix $P > 0$.  The probability that $Y$ started from $0$ hits
  $\boundarylines$ before $\farpoint$ is $O(\frac{1}{\log\epsilon})$.
\end{lemma}

\newcommand{\Oe}{O(\epsilon)}

\begin{proof}
  $Y$ consists of a sequence of excursions, each of which satisfies
  exactly one of the following conditions
  \begin{itemize}
  \item the excursion hits $\boundarylines$ (with probability
    $O(\epsilon$))
  \item the excursion does not hit $\boundarylines$ but does hit
    $\farpoint$ (with probability $\Omegaeloge$)
  \item the excursion does not hit $\boundarylines$ or $\farpoint$ before
    returning to $\origin$
  \end{itemize}
  When $Y$ returns to $\origin$ a new excursion begins independent of
  the previous excursions.  Thus the probability that the first
  condition occurs before the second is exactly the probability of the
  first as fraction of the sum of their probabilities, that is
  \[
  \frac{\Oe}{\Omegaeloge + \Oe} = O\left(\frac{1}{\log\epsilon}\right)
  \]
\end{proof}

We can choose $\farpoint$ arbitrarily.  The following statements hold
with probability approaching $1$ as $\epsilon \to 0$.

By making $\epsilon$ small we can force $\rotproc$ to travel to a far
away point on the horizontal axis before hitting $\boundarylines$.
Once $\rotproc$ has travelled far away it will take time greater that
$1$ to return to the origin.  Thus during the whole of the time
interval $[0,1]$ it remains within the strip bounded by the lines
$\boundarylines$.

Formally

\begin{lemma}
  The probability that before time $1$, $Y$ has hit $\boundarylines$
  is $o(1)$.
\end{lemma}

\begin{proof}
  Fix $\eta > 0$.

  Choose $P$ so that the probability that standard Brownian motion
  travels from $0$ to $P$ in time less than $1$ is less than
  $\eta$.

  Choose $\epsilon$ such that the probability that $Y$ hits
  $\boundarylines$ before $\farpoint$ is less than $\eta$.

  Then the probability that $Y$ hits $\boundarylines$ before $\farpoint$
  or takes less time than $1$ to return from $\farpoint$ to $\origin$ is
  less than $2\eta$.
\end{proof}

This proves the following.

\begin{lemma}
  Write $(Y^1, Y^2) = Y$.  Then $\P(\sup_{t \in [0,1]} |Y^2_t| \ge
  \delta) \to 0$ as $\epsilon \to 0$.
\end{lemma}

In other words,

\begin{cor}
  $\P(\sup_{t \in [0,1]} |\sampled_t - \resamplede_t | \ge \delta) \to
  0$ as $\epsilon \to 0$.
\end{cor}

And by the scale-invariance properties of Brownian motion, proving the
convergence uniformly on the interval $[0,1]$ is as good as proving it
uniformly for every bounded interval.  Thus this suffices to prove
Lemma \ref{lem:resamplede-to-sampled}\label{proof-of-lem:resamplede-to-sampled}, i.e.\ that \statementoflemresampledetosampled.

We conclude the section by providing the proofs of the two previously
stated lemmas.

\begin{proof}[Proof of Lemma \ref{lem:Phitboundaryline}]
Fix $\delta > 0$.  Consider the process $\rotproc$ run until either
it hits $A$ or is absorbed.  We will show that the probability it
hits $A$ is $O(\epsilon)$.

The only difficulty is that $\Le$ is sometimes absorbing and
sometimes not.

We analyse $\rotproc$ through the following state machine.

Consider $\rotproc = (x,0) \in \Le$ (for some $x$) and $\Le$ being in
a non-absorbing state.  We will call this the start state.  From the
start state, $Y$ will eventually reach one of two mutually exclusive
states, according to which of the following occurs first

\newcommand{\intermediatelines}{I}

\begin{itemize}
\item $Y$ is absorbed
\item $Y$ hits $\intermediatelines$
\end{itemize}

The probability of the first is bounded below by some constant $C > 0$
which does not depend on $x$ and $\epsilon$, and regardless of the
history of the state machine.

\newcommand{\stateintermediate}{Intermediate}

If the second happens we will say that the state machine goes into
state $\stateintermediate$.  From this state, the possible transitions
are according to which of the following occurs first

\begin{itemize}
\item $Y$ is absorbed
\item $Y$ hits $\boundarylines$
\item $Y$ hits $\Le$ when it is in a non-absorbing state
\end{itemize}

When the last happens we have returned to the start state.

The probability of the second transition is exactly equal to
$\epsilon/\delta$ regardless of the history of the state machine.

From this description it is straightforward to deduce that the
probability that the machine started when $Y = \origin$ (or indeed when
$Y = (x, 0)$ for any $x \in \Le$ and $\Le$ not absorbing) will hit
$\boundarylines$ before being absorbed is
bounded above by $\epsilon/\delta C$, so indeed this probability is
$O(\epsilon)$.
\end{proof}

\begin{proof}[Proof of Lemma \ref{lem:Pabsorbedandtravelsfar}]
From $Y = \origin$ there is a positive probabilty, $C$ say, not
depending on $\epsilon$, that $Y$ hits $Q = [0, \epsilon] \cross
\{\epsilon\}$ without having been absorbed.

From $Y \in Q$ the hitting density on $L^+$ is at least $K
\frac{1}{\epsilon} \frac{1}{1 + (y/\epsilon)^2} dy$, where $K$ is a
normalising constant independent of $\epsilon$.

So the probability that $Y$ started from some point in $Q$ hits $L^+$
between $\epsilon$ and $1$ and then travels to a point $\farpoint$ is at least
\[
\frac{K}{P} \int_{\epsilon}^{1} \frac{y/\epsilon}{1 + (y/\epsilon)^2}
\, dy
\]
which is
\[
\frac{K\epsilon}{2P} \log\left(\frac{1 + (1/\epsilon)^2}{2}\right)
\]
which is $\Omegaeloge$.
\end{proof}
}
