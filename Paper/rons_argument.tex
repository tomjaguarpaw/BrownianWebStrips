{
\section{Ron's argument}

\newcommand{\bandwidth}{\delta}
\newcommand{\rotproc}{Y}

\newcommand{\union}{\cup}
\renewcommand{\L}{L^+ \union L^-}
\newcommand{\Le}{L^\epsilon}

\subsection{Probability of escape}

It will be slightly easier to work with the process $\rotproc =
(\sampled + \resamplede, \sampled - \resamplede)$ which is $(\sampled,
\resamplede)$ rotated clockwise by $\pi / 4$ radians, and enlarged by a factor of
$\sqrt{2}$.

\newcommand{\boundarylines}{A}

\begin{definition}
  An excursion of $Y$ is two stopping times $S \le T$ such that $Y_S =
  (0,0)$ and $T = \inf\{ t > S : Y_t = (0,0) \}$.
\end{definition}

Note that there is a.s.\ no excursion for which $S = T$, since almost
surely $Y$ spends a positive amount of time away from $(0,0)$ before
returning.  We can then consider $Y$ as a process which performs a
sequence of excursions.

\begin{definition}
  We say $Y$ is \emph{absorbed} when it hits $\L$, or it hits $\Le$
  when it is in an absorbing state.
\end{definition}

\begin{lemma}
  \label{lem:Phitboundaryline}
  The probability that $Y$ hits $\boundarylines$ during an excursion
  is $O(\epsilon)$.
\end{lemma}

\newcommand{\Omegaeloge}{\Omega(\epsilon\log\epsilon)}

\begin{lemma}
  \label{lem:Pabsorbedandtravelsfar}
  Fix $P > 0$.  The probability that $Y$ hits $(P,0)$ during an
  excursion is $\Omegaeloge$.
\end{lemma}

\begin{lemma}
  Fix $P > 0$.  The probability that $Y$ hits $(P,0)$ during an
  excursion without hitting $A$ is $\Omegaeloge$.
\end{lemma}

\begin{proof}
  By Lemmas \ref{lem:Phitboundaryline} and
  \ref{lem:Pabsorbedandtravelsfar} the probability is at least
  $\Omegaeloge - O(\epsilon) = \Omegaeloge$.
\end{proof}

\begin{lemma}
  Fix $P > 0$.  The probability that $Y$ started from $0$ hits
  $\boundarylines$ before $(P,0)$ is $O(\frac{1}{\log\epsilon})$.
\end{lemma}

\newcommand{\Oe}{O(\epsilon)}

\begin{proof}
  $Y$ consists of a sequence of excursions, each of which satisfies
  exactly one of the following conditions
  \begin{itemize}
  \item the excursion hits $\boundarylines$ (with probability
    $O(\epsilon$))
  \item the excursion does not hit $\boundarylines$ but does hit
    $(P,0)$ (with probability $\Omegaeloge$)
  \item the excursion does not hit $\boundarylines$ or $(P,0)$ before
    returning to $(0,0)$
  \end{itemize}
  When $Y$ returns to $(0,0)$ a new excursion begins independent of
  the previous excursions.  Thus the probability that the first
  condition occurs before the second is exactly the probability of the
  first as fraction of the sum of their probabilities, that is
  \[
  \frac{\Oe}{\Omegaeloge + \Oe} = O\left(\frac{1}{\log\epsilon}\right)
  \]
\end{proof}

\begin{lemma}
  The probability that before time $1$, $Y$ has hit $\boundarylines$
  is $o(1)$.
\end{lemma}

\begin{proof}
  Fix $\eta > 0$.

  Choose $P$ so that the probability that standard Brownian motion
  travels from $0$ to $P$ in time less than $1$ is less than
  $\eta$.

  Choose $\epsilon$ such that the probability that $Y$ hits
  $\boundarylines$ before $(P,0)$ is less than $\eta$.

  Then the probability that $Y$ hits $\boundarylines$ before $(P,0)$
  or takes less time than $1$ to return from $(P,0)$ to $(0,0)$ is
  less than $2\eta$.
\end{proof}

\begin{lemma}
  Write $(Y^1, Y^2) = Y$.  Then $\P(\sup_{t \in [0,1]} |Y^2_t| \ge
  \delta) \to 0$ as $\epsilon \to 0$.
\end{lemma}

\begin{lemma}
  $\P(\sup_{t \in [0,1]} |\sampled_t - \resamplede_t | \ge \delta) \to
  0$ as $\epsilon \to 0$.
\end{lemma}

\begin{proof}[Proof of Lemma \ref{lem:Phitboundaryline}]
Fix $\delta > 0$.  Consider the process $\rotproc$ run until either
it hits $A$ or is absorbed.  We will show that the probability it
hits $A$ is $O(\epsilon)$.

The only difficulty is that $\Le$ is sometimes absorbing and
sometimes not.

We analyse $\rotproc$ through the following state machine.

Consider $\rotproc = (x,0) \in \Le$ (for some $x$) and $\Le$ being in
a non-absorbing state.  We will call this the start state.  From the
start state, $Y$ will eventually reach one of two mutually exclusive
states, according to which of the following occurs first

\newcommand{\intermediatelines}{I}

\begin{itemize}
\item $Y$ is absorbed
\item $Y$ hits $\intermediatelines$
\end{itemize}

The probability of the first is bounded below by some constant $C > 0$
which does not depend on $x$ and $\epsilon$, and regardless of the
history of the state machine.

\newcommand{\stateintermediate}{Intermediate}

If the second happens we will say that the state machine goes into
state $\stateintermediate$.  From this state, the possible transitions
are according to which of the following occurs first

\begin{itemize}
\item $Y$ is absorbed
\item $Y$ hits $\boundarylines$
\item $Y$ hits $\Le$ when it is in a non-absorbing state
\end{itemize}

When the last happens we have returned to the start state.

The probability of the second transition is exactly equal to
$\epsilon/\delta$ regardless of the history of the state machine.

From this description it is straightforward to deduce that the
probability that the machine started when $Y = (0,0)$ (or indeed when
$Y = (x, 0)$ for any $x \in \Le$ and $\Le$ not absorbing) will hit
$\boundarylines$ before being absorbed is
bounded above by $\epsilon/\delta C$, so indeed this probability is
$O(\epsilon)$.
\end{proof}

\begin{proof}[Proof of Lemma \ref{lem:Pabsorbedandtravelsfar}]
From $Y = (0,0)$ there is a positive probabilty, $C$ say, not
depending on $\epsilon$, that $Y$ hits $Q = [0, \epsilon] \cross
\{\epsilon\}$ without having been absorbed.

From $Y \in Q$ the hitting density on $L^+$ is at least $K
\frac{1}{\epsilon} \frac{1}{1 + (y/\epsilon)^2} dy$, where $K$ is a
normalising constant.

So the probability that $Y$ started from some point in $Q$ hits $L^+$
between $\epsilon$ and $1$ and then travels to a point $(P,0)$ is at least
\[
\frac{K}{P} \int_{\epsilon}^{1} \frac{y/\epsilon}{1 + (y/\epsilon)^2}
\, dy
\]
which is
\[
\frac{K\epsilon}{2P} \log\left(\frac{1 + (1/\epsilon)^2}{2}\right)
\]
which is $\Omegaeloge$.
\end{proof}

\section{From single point split to joining strips}

\newcommand{\AuB}{A \cup B}

\TODO{Consider two horizontal strips $A$ and $B$ whose union is another
horizontal strip $\AuB$}{Explain this better}.

\newcommand{\F}{\mathcal{F}}

$\sampled$ is $\F$-measurable.

\newcommand{\sampledT}{\sampled^T}
\newcommand{\sampledTe}{\sampled^{T,\epsilon}}
\newcommand{\resampledeT}{{\resamplede}^{,T}}

Let $\sampledT$ be the process $\sampled$ truncated on leaving
$\AuB$.  Processes of this form generate $\F_{\AuB}$.

$\sampledTe$ is the $\epsilon$-resampled truncated process and is
$\F_A \tensor \F_B$-measurable.

$\resampledeT$ is $\resamplede$ truncated to $\AuB$

$\resamplede$ is $\sampled$ resampled on an $\epsilon$ strip.

\begin{obs}
  Processes of the form $\sampledT$ are $\F_A \tensor
  \F_B$-measurable.
\end{obs}

\begin{lemma}
  $\resampledeT = \sampledTe$
\end{lemma}

\begin{lemma}
  $\sampledTe \to \sampledT$ as $\epsilon \to 0$
\end{lemma}

\begin{proof}
  We could show this directly along the lines above.  However instead
  we will do it as follows.

  Note that
  \begin{itemize}
  \item $\resamplede \to^\P \sampled$
  \item the probability that $f \mapsto \restrict{I}{f}$ is continuous
    at $X$ is $1$
  \end{itemize}
  so we conclude that $\restrict{I}{\resamplede} \to^\P
  \restrict{I}{\sampled}$ by the continuous mapping theorem (see for
  example \cite{billingsley}, p. 21, Theorem 2.7). Note further that
  \begin{itemize}
  \item $\resampledeT = \restrict{I}{\resamplede}$
  \item $\sampledT = \restrict{I}{\sampled}$
  \end{itemize}
  so that $\resampledeT \to^\P \sampledT$.
\end{proof}

\begin{lemma}
  Recall the definition of $\restrict{\cdot}{\cdot}$ from Definition
  \ref{def:restrict}.  If $\stripleavetime{f}$ is not a turning point of the path $f$,
  then the map $f \mapsto \restrict{I}{f}$ is continuous at $f$ in the
  topology of uniform convergence on compacts.
\end{lemma}

\begin{proof}
  \DOTHIS{We sketch a proof of this straightforward result in
    classical analysis.

    \newcommand{\fn}{f_n}

    \newcommand{\T}{\stripleavetime{f}}
    \newcommand{\Tn}{\stripleavetime{\fn}}

    Suppose $\fn \to f$ uniformly on compacts.  Then since $\T$ is not
    a turning point of $f$ it is easily seen that $\Tn \to \T$ and
    that eventually $\fn(\Tn) = f(\T)$.

    \renewcommand{\d}{\delta}
    \newcommand{\e}{\epsilon}
    \newcommand{\starttime}{-}

    Since $f$ is continuous, choose $\d$ such that
    $|f(\T) - f(x)| \le \epsilon$ when
    $|\T - x| \le \delta$.  Eventually
    \begin{itemize}
    \item $\fn(\Tn) = f(\T)$
    \item $|\Tn - \T| \le \d$
    \item $|\fn - f| \le \e$ uniformly on some compact containing
      $[\starttime, \T+\d]$.
    \end{itemize}

    \newcommand{\stoppedfn}{\restrict{I}{\fn}}
    \newcommand{\stoppedf}{\restrict{I}{f}}

    \newcommand{\condition}[2]{for $t \in {#1}$ we have
      $|\stoppedfn(t) - \stoppedf(t)| #2$}
    so
    \begin{itemize}
    \item \condition{[\starttime, \T-\d)}{\le \e}
    \item \condition{[\T-\d, \T+\d]}{\le 2\e}
    \item \condition{(\T+\d, \infty)}{= 0}
    \end{itemize}
    so indeed $\stoppedfn \to \stoppedf$ uniformly on compacts (indeed
    globally since they are both eventually $0$ outside some
    compact).}{The only tricky part is the second item.  There are
      four cases to consider.  The worst case is when $\stoppedf$ has
      been truncated and $\stoppedfn$ hasn't.  But $\stoppedf$ is
      equal to the level, and so is within $\e$ of $f$.  OTOH
      $\stoppedfn = \fn$ so is within $\e$ of $f$.  Thus the $2\e$
      bound.}
\end{proof}
}
