{
\section{Ron's argument}

\newcommand{\bandwidth}{\delta}
\newcommand{\rotproc}{Y}

\newcommand{\union}{\cup}
\renewcommand{\L}{L^+ \union L^-}
\newcommand{\Le}{L^\epsilon}

\subsection{Probability of escape}

It will be slightly easier to work with the process $\rotproc =
(\sampled + \resamplede, \sampled - \resamplede)$ which is $(\sampled,
\resamplede)$ rotated by $\pi / 4$ radians, and scaled by a factor of
$\sqrt{2}$.

\newcommand{\boundarylines}{A}

\begin{definition}
  An excursion of $Y$ is two stopping times $S < T$ such that $Y_S =
  (0,0)$ and $T = \inf\{ t \ge S : Y_t = (0,0) \}$.  \FIXME{Note that
    this is well defined since almost surely $Y$ spends a positive
    amount of time away from $(0,0)$ before returning.}{This isn't
    quite the right way to explain this}
\end{definition}

\begin{lemma}
  \label{lem:Phitboundaryline}
  The probability that $Y$ hits $\boundarylines$ during an excursion
  is $O(\epsilon)$.
\end{lemma}

\newcommand{\Omegaeloge}{\Omega(\epsilon\log\epsilon)}

\begin{lemma}
  \label{lem:Pabsorbedandtravelsfar}
  Fix $P > 0$.  The probability that $Y$ hits $(P,0)$ during an
  excursion is $\Omegaeloge$.
\end{lemma}

\begin{lemma}
  Fix $P > 0$.  The probability that $Y$ hits $(P,0)$ during an
  excursion without hitting $A$ is $\Omegaeloge$.
\end{lemma}

\begin{proof}
  By Lemmas \ref{lem:Phitboundaryline} and
  \ref{lem:Pabsorbedandtravelsfar} the probability is at least
  $\Omegaeloge - O(\epsilon) = \Omegaeloge$.
\end{proof}

\begin{lemma}
  Fix $P > 0$.  The probability that $Y$ started from $0$ hits $(P,0)$
  before $\boundarylines$
\end{lemma}

\begin{proof}
  $Y$ consists of a sequence of excursions, each of which satisfies
  exactly one of the following conditions
  \begin{itemize}
  \item the excursion hits $\boundarylines$
  \item the excursion does not hit $\boundarylines$ but does hit
    $(P,0)$
  \item 
  \end{itemize}
\end{proof}

\begin{lemma}
  The probability that before time $1$, $Y$ has hit $\boundarylines$
  is $O(\log\epsilon)$.
\end{lemma}

Fix $\delta > 0$.  Consider the process $\rotproc$ run until it hits
either $A$ or $\L$.  We will show that the probability it
hits $A$ first is $O(\epsilon)$.

The only difficulty is that $\Le$ is sometimes absorbing and
sometimes not.

We analyse $\rotproc$ through the following state machine.

Consider $\rotproc = (x,0) \in \Le$ (for some $x$) and $\Le$ being in
a non-absorbing state.  We will call this the start state.  From the
start state, $Y$ will eventually reach one of three mutually exclusive
states, according to which of the following occurs first

\newcommand{\intermediatelines}{I}

\begin{itemize}
\item $Y$ hits $\L$
\item $Y$ hits $\intermediatelines$
\item $Y$ hits $\Le$ when it is in an absorbing state
\end{itemize}

The probability of the first is bounded by some constant $C$, say,
which does not depend on $x$ and $\epsilon$, and regardless of the
history of the state machine.

\newcommand{\stateintermediate}{Intermediate}

If the second happens we will say that the state machine goes into
state $\stateintermediate$.  From this state, the possible transitions
are according to which of the following occurs first

\begin{itemize}
\item $Y$ hits $\L$
\item $Y$ hits $\boundarylines$
\item $Y$ hits $\Le$ when it is in an absorbing state
\item $Y$ hits $\Le$ when it is in a non-absorbing state
\end{itemize}

When the last happens we have returned to the start state.

The probability of the second transition is exactly equal to
$\epsilon/\delta$ regardless of the history of the state machine.

From this description it is straightforward to deduce that the
probability that the machine started when $Y = (0,0)$ (or indeed when
$Y = (x, 0)$ for any $x \in \Le$ and $\Le$ not absorbing) will hit
$\boundarylines$ before being absorbed on $L^+ \union \Le \union L^-$ is
bounded above by $\epsilon/\delta C$, so indeed this probability is
$O(\epsilon)$.

\subsection{Probability of capture}

From $Y = (0,0)$ there is a positive probabilty, $C$ say, not
depending on $\epsilon$, that $Y$ hits $Q = [0, \epsilon] \cross
\{\epsilon\}$ without having been absorbed on $L^+ \union \Le \union
L^-$.

From $Y \in Q$ the hitting density on $L^+$ is at least $K
\frac{1}{\epsilon} \frac{1}{1 + (y/\epsilon)^2} dy$, where $K$ is a
normalising constant.

So the probability that $Y$ started from some point in $Q$ hits $L^+$
between $\epsilon$ and $1$ and then travels to a point $(P,0)$ is at least
\[
\frac{K}{P} \int_{\epsilon}^{1} \frac{y/\epsilon}{1 + (y/\epsilon)^2}
\, dy
\]
which is
\[
\frac{K\epsilon}{2P} \log\left(\frac{1 + (1/\epsilon)^2}{2}\right)
\]
which is $\Omegaeloge$.

\subsection{Probability of escape before capture}

So the probability that $Y$ reaches $(P,0)$ before hitting
$\boundarylines$ in one launch is at least
$\Omegaeloge - O(\epsilon) = \Omegaeloge
$.

Thus the probability that $Y$ started from $(0,0)$ reaches
$\boundarylines$ before $(P,0)$ is
\[
\frac{O(\epsilon)}{\Omegaeloge - O(\epsilon)} = O(\log\epsilon)
\]

}
