\documentclass[11pt]{article}
\usepackage{amsmath, amssymb, amsfonts, amsthm,latexsym, hyperref}
\usepackage[all,cmtip]{xy}
\usepackage{verbatim}
%\baselineskip = 18pt

\oddsidemargin  0pt     %   Left margin on odd-numbered pages.
\evensidemargin 0pt     %   Left margin on even-numbered pages.
\marginparwidth 40pt    %   Width of marginal notes.
\marginparsep 10pt      % Horizontal space between outer margin and
                        % marginal note

% VERTICAL SPACING:
\topmargin 0pt           % Nominal distance from top of page to top of
                         %    box containing running head.
\headsep 10pt            %    Space between running head and text.

% DIMENSION OF TEXT:

\textheight 8.5in        %Height of text(including footnotes and figures,
                         % excluding running head and foot).
\textwidth 6.6in         % Width of text line.
\renewcommand{\baselinestretch}{1.2}
\topmargin 0pt \headsep 0pt


\bibliographystyle{plain}
\newtheorem*{uthm}{Theorem}
\newtheorem{thm}{Theorem}
\newtheorem{theorem}{Theorem}[section]
\newtheorem{claim}{Claim}
\newtheorem{obs}{Observation}
\newtheorem{rem}{Remark}

\newtheorem{lem}{Lemma}[section]
\newtheorem{lemma}{Lemma}[section]

\newtheorem{propos}{Proposition}


\newtheorem{defin}{Definition}[section]
\newtheorem{cor}{Corollary}
\newtheorem{conjecture}{Conjecture}
\newtheorem{problem}{Problem}
\begin{document}
{
\section{Suitable title}
Let $\xi_\epsilon=(X,Y,\epsilon)$ be a couple of $\epsilon$ resampled Brownian webs as defined in ?????. We denoted by $e_\epsilon(t)$ a sample of $\xi_\epsilon$ starting at some given point at time $0$. $x(t)$ will denote a sample of $X$ starting at the same point. Our goal in this section is to prove the following theorem:

\begin{theorem}\label{thm:main1}
under the previous notations, $\lim_{\epsilon \rightarrow 0} |e_\epsilon(1)- x(1)| = 0$ where the convergence is in probability.
\end{theorem}

In order to prove this it will be easier to prove the following stronger claim:
\begin{propos}\label{prop:main2}
For every $t\in[0,1]$ and every $\delta>0$ it is true that $lim_{\epsilon \rightarrow 0}P\left(\text{either } x(t)=e_\epsilon(t) \text{ or } x(t)^2+(e_\epsilon(t)-x(t))^2<\delta\right)=1$. Equivalently the two dimensional process 
$(x(t),(e_\epsilon(t)-x(t)))$ a.s. leaves the $\sqrt\delta$ ball only through $(\sqrt\delta,0)$ or through $(-\sqrt\delta,0)$.
\end{propos}

One can easily see from the definition of $\epsilon$ resampled Brownian webs that proposition \ref{prop:main2} infers theorem \ref{thm:main1}.
We begin by presenting a handy theorem whose proof we delay to subsection \ref{sec:conformal map}:

\begin{theorem}\label{thm:no-escape}
under the previous notations, suppose that on time $t_0$ we have $e(t_0)=0$, let $t>0$ be the first time at which $x(t)^2+\frac{(x(t)-(e_\epsilon(t)-x(t))^2}2=1$, then the probability that $x(t)=e_\epsilon(t)=\pm1$ is at least $1-O(\frac1{log\epsilon})$.
\end{theorem}

Applying scaling invariance to this theorem immediately yields the following corollary:
\begin{cor}\label{cor:cor0}
let $\delta>0$ suppose that on time $t_0$ we have $e(t_0)=(0,0)$, let $t>0$ be the first time at which $x(t)^2+\frac{(x(t)-(e_\epsilon(t)-x(t))^2}2=\delta$, then the probability that $x(t)=e_\epsilon(t)=\pm\sqrt\delta$ is at least $1-O(\frac{1}{log\epsilon/\delta})$.
\end{cor}

It will, however be more comfortable to use the following weaker result:
\begin{cor}\label{cor:cor1}
let $\delta>0$ suppose that on time $t_0$ we have $e(t_0)=(0,0)$, let $t>0$ be the first time at which $x(t)^2+(x(t)-(e_\epsilon(t)-x(t))^2=\delta$, then the probability that $x(t)=e_\epsilon(t)=\pm\sqrt\delta$ is at least $1-O(\frac{1}{log\epsilon/\delta})$.
\end{cor}

In order to overcome the gap between corollary \ref{cor:cor1} and theorem \ref{thm:main1} we present the following property of Browning motion:
\begin{propos}\label{prop:prop1}
let $\delta,\zeta>0$ and let $x(t)$ be a one dimensional browning motion such that $x(0)=\delta$. there exists some $k$ such that $P(\min_t(x(t)=0)<\frac1k)<\zeta$.
\end{propos}

We are now ready to give a proof to proposition \ref{prop:main2}.

\begin{proof}[proof (proposition \ref{prop:main2})]
Let $\zeta,\delta>0$. We find $\epsilon>0$ such that $P(|e_\epsilon(1)- x(1)|\le\sqrt\delta)<\zeta$. Let $t_1,...,t_k$ be the set of points such that $x(t_i)=0$,
and there exists a time $t_i<t'_i<t_i+1$ such that $x(t'_i)=\pm\sqrt\delta$. By proposition \ref{prop:prop1}, there exists $K\in\mathbb{N}$ such that with probability
$\sqrt\zeta$ it is true that $\sum_{i=1}^k(t_{i+1}-t'_i)\ge1/K$, and thus $k\le K$. If there exists $s_i\in (t_i, t'_i)$ for which $x(s_i)^2+(e_\epsilon(s_i)-x(s_i))^2=\delta$ then by the definition of $t'_i$, $x(s_i)<\sqrt\delta$. By choosing $\epsilon$ properly in corollary \ref{cor:cor1}  this happens with probability at most $\sqrt[2K]{\zeta}$. By the independence the process on the  intervals $[t_i,t'_i]$ we can multiply those probabilities, getting that the probability that for any $s_i$ we have $x(s_i)^2+(e_\epsilon(s_i)-x(s_i))^2=\delta$ is less then $\sqrt\zeta$. Since those intervals are not intersecting with the intervals $(t'_i,t_{i+1})$ we have for the chosen epsilon $P(\exists t | x(t)^2+(e_\epsilon(t)-x(t))^2=\delta, x(t)\neq \sqrt \delta)<\zeta$ for all $t\in(0,1)$.
\end{proof}

\subsection{Conformal maps}\label{sec:conformal map}
We remind the reader our notation. Let $\xi_\epsilon=(X,Y,\epsilon)$ be a couple of $\epsilon$ resampled Brownian webs as defined in ?????. We denoted by by $e_\epsilon(t)$ a sample of $\xi_\epsilon$ starting at some given point at time $0$. We also denote by $x(t)$ a sample of $X$ starting at the same point.
In this section we prove theorem \ref{thm:no-escape}. Our main technical tool in doing so will be conformal maps. We begin by introducing a few more definitions:
Let $y(t)=\frac{e_\epsilon(t)-x(t)}{\sqrt2}$, and let $\bar{b}(t)=(x(t),y(t))$. We can now rephrase theorem \ref{thm:no-escape} as:

\begin{propos}[Theorem \ref{thm:no-escape} rephrased]\label{prop:reph}
under the previous notations, suppose that on time $t_0$ we have $\bar{b}(0)=(0,0)$, let $t_1>0$ be the first time at which $||b(t_1)||_2=1$, then the probability that $y(t_1)\neq0$ is less then $O(\frac1{log\epsilon})$.
\end{propos}

We define the following indicator function $A(t):=(y(t)=0)\wedge(|x(t)|>\epsilon)$. Note that whenever $A(t)=1$, we can regard $b(t)$ as a one dimensional Brownian motion on the $x$-axis, while whenever $A(t)=0$ we can regard it as a two dimensional Brownian motion. It will be comfortable to treat the process as a 4-states 


\end{document}

