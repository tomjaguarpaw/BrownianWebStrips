{
\section{Remarks and open problems}

\label{sec:open-problems}

In this paper we present the second known example of a two-dimensional
black noise, after the scaling limit of critical planar percolation,
as proved by Schramm and Smirnov \cite{schramm-smirnov}.  They also remark
that \sigfield{}s can be associated to a larger class of domains than
just rectangles in a way that still allows the \sigfield{} of a larger
domain to be recovered from the \sigfield{}s of two smaller domains
that partition it.
In particular, in Remark 1.8 they claim that this can be done for the
scaling limit of site percolation on the triangular lattice, as long
as border between those domains has Hausdorff dimension less
than 5/4, and cannot be done if the border has Hausdorff dimension
greater then than 5/4. This raises the following question:

\begin{openproblem}
  \label{openproblem:extend}
  To what class of two-dimensional domains can the Brownian web noise be extended?
\end{openproblem}

We expect the answer to be more sophisticated than for percolation,
since the Brownian web is not rotationally invariant.  
This suggests that Hausdorff dimension is not a sufficient measurement
to determine from which subdomains the Brownian web can be
reconstructed.  In some sense it is easier to reconstruct the Brownian
web from vertical strips than it is from horizontal strips.

We may obtain a better understanding of Open Problem
\ref{openproblem:extend} if we can answer

\begin{openproblem}
  Give an explicit example of domains to which the noise
  cannot be extended.
\end{openproblem}

By general results, such domains must exist (see
\cite{tsirelson-nonclassical-stochastic-flows} Theorem 11a2 and
Section 11b).

Moreover, having seen that the Brownian web is a two-dimensional black
noise, further examples in two dimensions (and indeed in higher
dimensions) are called for.  Their
discovery would hopefully shed light on the nature of black noises.

\begin{openproblem}
  Find more examples of two-dimensional black noises.  Show an example
  of a black noise in three dimensions or higher.
\end{openproblem}

Readers may wish to note that in
\cite{tsirelson-noise-as-a-boolean-algebra} Tsirelson extends the
concept of a noise to a much more abstract and general setting.  This
allows results on noises to be formulated and proved without having an
explicit underlying geometric base.  However, our methods here which are
concrete and geometric in nature are more conveniently described in terms
of the earlier formulation.
}
