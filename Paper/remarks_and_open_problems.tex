{
\section{Remarks and open problems}

\label{sec:open-problems}

In this paper we present the second known example of a two-dimensional
black noise, after the scaling limit of critical planar percolation,
as proved by Schramm and Smirnov \cite{schramm-smirnov}.  They also remark
that \sigfield{}s can be associated to a larger class of domains than
just rectangles in a way that still allows the \sigfield{} of a larger
domain to be recovered from the \sigfield{}s of two smaller domains
that partition it.
Here we call this procedure ``extending the noise''.
In particular, in Remark 1.8 the authors refer to the work of Garban,
Pete and Schramm (Remark 8.5 of \cite{garban-pete-schramm}) in
mentioning that this can be done for the
scaling limit of site percolation on the triangular lattice, as long
as border between those domains has Hausdorff dimension less
than 5/4, and cannot be done if the border has Hausdorff dimension
greater then than 5/4.
Consequently, the noise of the scaling limit of site percolation on
the triangular lattice can be extended to the class of all domains
whose boundaries have Hausdorff dimension less than 5/4.
This raises the following question:

\begin{openproblem}
  \label{openproblem:extend}
  To what class of two-dimensional domains can the Brownian web noise be extended?
\end{openproblem}

We expect the answer to be more sophisticated than for percolation,
since the Brownian web is not rotationally invariant.
This suggests that Hausdorff dimension is not a sufficient measurement
to determine from which subdomains the Brownian web can be
reconstructed.  In some sense it is easier to reconstruct the Brownian
web from vertical strips than it is from horizontal strips.

Given that we now have a second example of a two dimensional black
noise, one may also ask the following:

\begin{openproblem}
  Are there black noises in three dimensions and higher?
\end{openproblem}

Readers may wish to note that in
\cite{tsirelson-noise-as-a-boolean-algebra} Tsirelson extends the
concept of a noise to a much more abstract and general setting.  This
allows results on noises to be formulated and proved without having an
explicit underlying geometric base.  However, our methods here which are
concrete and geometric in nature are more conveniently described in terms
of the earlier formulation.
}
