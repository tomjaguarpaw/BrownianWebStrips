{
\section{Remarks and open problems}

\label{sec:open-problems}

In this paper we present the second known example of a two-dimesional
black noise, after the scaling limit of critical planar percolation,
as proved by Schramm and Smirnov.  In their result they also proved
that \sigfield{}s can be associated to a larger class of domains than
just rectangles in a way that still allows the \sigfield{} of a larger
domain to be recovered from the \sigfield{}s of two smaller domains
that partition it, as long as border between those domains has
Hausdorff dimension strictly less than 5/4.
Additionally they show that this cannot be done when the border has
dimension strictly greater than 5/4.
This raises the following question:

\begin{openproblem}
  \label{openproblem:extend}
  To what class of domains can the Brownian web noise be extended?
\end{openproblem}

We suspect the answer to be more sophisticated than for percolation,
since the Brownian web is not rotationally invariant.  
This suggests that Hausdorff dimension is not a sufficient measurment
to determine from which subdomains the Brownian web can be
reconstructed.  In some sense it is easier to reconstruct the Brownian
web from vertical strips than it is from horizontal strips.

We may obtain a better understanding of Open problem
\ref{openproblem:extend} if we can answer

\begin{openproblem}
  Can we demonstrate an explicit example of domains to which the noise
  cannot be extended?
\end{openproblem}

By general results, such domains must exist (see
\cite{tsirelson-nonclassical-stochastic-flows} Theorem 11a2 and
Section 11b).

Moreover, having seen that the Brownian web is a two-dimensional black
noise, further examples in two dimensions (and indeed in higher
dimensions) are called for.  Their
discovery would shed light on the nature of black noises.

\begin{openproblem}
  Find more examples of two-dimensional black noises.  Show an example
  of a black noise in three dimensions or more.
\end{openproblem}
}
