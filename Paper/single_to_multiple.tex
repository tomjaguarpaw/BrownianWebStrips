{
\section{From single point split to joining strips}

\newcommand{\Res}[1]{\mathcal{R}(#1)}

Recall the definition of $\restrict{\cdot}{\cdot}{\cdot}$ from
Definition \ref{def:restrict}.  Just in this section, for notational
convenience, for fixed $a \le b$ we will write $\Res{\cdot}$ for
$\restrict{a}{b}{\cdot}$.

\newcommand{\AuB}{A \cup B}

\TODO{Consider two horizontal strips $A$ and $B$ whose union is another
horizontal strip $\AuB$}{Explain this better}.

\newcommand{\F}{\mathcal{F}}

$\sampled$ is $\F$-measurable.

\newcommand{\sampledT}{\sampled^T}
\newcommand{\sampledTe}{\sampled^{T,\epsilon}}
\newcommand{\resampledeT}{{\resamplede}^{,T}}

Let $\sampledT$ be the process $\sampled$ truncated on leaving
$\AuB$.  Processes of this form generate $\F_{\AuB}$.

$\sampledTe$ is the $\epsilon$-resampled truncated process and is
$\F_A \tensor \F_B$-measurable.

$\resampledeT$ is $\resamplede$ truncated to $\AuB$

$\resamplede$ is $\sampled$ resampled on an $\epsilon$ strip.

\begin{obs}
  Processes of the form $\sampledT$ are $\F_A \tensor
  \F_B$-measurable.
\end{obs}

\begin{lemma}
  $\resampledeT = \sampledTe$
\end{lemma}

\begin{lemma}
  $\sampledTe \to \sampledT$ as $\epsilon \to 0$
\end{lemma}

\begin{proof}
  We could show this directly along the lines above.  However instead
  we will do it as follows.

  Note that
  \begin{itemize}
  \item $\resamplede \to^\P \sampled$
  \item the probability that $f \mapsto \Res{f}$ is continuous
    at $X$ is $1$
  \end{itemize}
  so we conclude that $\Res{\resamplede} \to^\P
  \Res{\sampled}$ by the continuous mapping theorem (see for
  example \cite{billingsley}, p. 21, Theorem 2.7). Note further that
  \begin{itemize}
  \item $\resampledeT = \Res{\resamplede}$
  \item $\sampledT = \Res{\sampled}$
  \end{itemize}
  so that $\resampledeT \to^\P \sampledT$.
\end{proof}

\begin{lemma}
  If $\stripleavetime{f}$ is not a turning point of the path $f$,
  then the map $f \mapsto \Res{f}$ is continuous at $f$ in the
  topology of uniform convergence on compacts.
\end{lemma}

\begin{proof}
  \DOTHIS{We sketch a proof of this straightforward result in
    classical analysis.

    \newcommand{\fn}{f_n}

    \newcommand{\T}{\stripleavetime{f}}
    \newcommand{\Tn}{\stripleavetime{\fn}}

    Suppose $\fn \to f$ uniformly on compacts.  Then since $\T$ is not
    a turning point of $f$ it is easily seen that $\Tn \to \T$ and
    that eventually $\fn(\Tn) = f(\T)$.

    \renewcommand{\d}{\delta}
    \newcommand{\e}{\epsilon}
    \newcommand{\starttime}{-}

    Since $f$ is continuous, choose $\d$ such that
    $|f(\T) - f(x)| \le \epsilon$ when
    $|\T - x| \le \delta$.  Eventually
    \begin{itemize}
    \item $\fn(\Tn) = f(\T)$
    \item $|\Tn - \T| \le \d$
    \item $|\fn - f| \le \e$ uniformly on some compact containing
      $[\starttime, \T+\d]$.
    \end{itemize}

    \newcommand{\stoppedfn}{\Res{\fn}}
    \newcommand{\stoppedf}{\Res{f}}

    \newcommand{\condition}[2]{for $t \in {#1}$ we have
      $|\stoppedfn(t) - \stoppedf(t)| #2$}
    so
    \begin{itemize}
    \item \condition{[\starttime, \T-\d)}{\le \e}
    \item \condition{[\T-\d, \T+\d]}{\le 2\e}
    \item \condition{(\T+\d, \infty)}{= 0}
    \end{itemize}
    so indeed $\stoppedfn \to \stoppedf$ uniformly on compacts (indeed
    globally since they are both eventually $0$ outside some
    compact).}{The only tricky part is the second item.  There are
      four cases to consider.  The worst case is when $\stoppedf$ has
      been truncated and $\stoppedfn$ hasn't.  But $\stoppedf$ is
      equal to the level, and so is within $\e$ of $f$.  OTOH
      $\stoppedfn = \fn$ so is within $\e$ of $f$.  Thus the $2\e$
      bound.}
\end{proof}
}
