\newcommand{\statementoflemresampledetosampled}{$\resamplede \to^\P \sampled$ as $\epsilon \to 0$}
{
\section{Recovering the Brownian web from its horizontal half-planes}

\label{recovering-from-half-planes}

\begin{notation}
Given a Brownian web $\webnoargs$, fix a starting point $(s,x)$ and write
$\sampled$ for the random process $t \mapsto \web{s}{t}{x}$
(which is a Brownian motion).
\end{notation}

\newcommand{\restrictupper}{\mathcal{R}_{+}}

We will define two $\sigma$-algebras.  The first, $\upperhp$,
represents the information contained in the trajectories of the
Brownian web in the upper half-plane.  It is generated by all the
trajectories of the Brownian web stopped when they leave the upper
half-plane.  Analogously we define $\lowerhp$ for the lower
half-plane.  We will use $\wholealgebra$ to denote the
$\sigma$-algebra generated by the whole web.

\begin{definition}
  For any path $f$ write $\restrictupper(f)$ for $f$ stopped at the
  first time it is outside the upper half-plane.  Then
  $\restrictupper(t \mapsto \web{s}{t}{x})$ is the trajectory of the
  web $\webnoargs$ started at point $(s,x)$ and stopped at the first
  time it is outside the upper half-plane.

  Define $\upperhp$ to be the $\sigma$-algebra generated by the
  collection of paths $\{\restrictupper(t \mapsto \web{s}{t}{x} : s,x
  \in \R \}$.

  Define $\lowerhp$ analogously.

  Write $\wholealgebra$ for the $\sigma$-algebra generated by the
  whole web, i.e.\ by all its trajectories $t \mapsto \web{s}{t}{x}$.
\end{definition}

We will show that, given a trajectory $\sampled$ of the Brownian web,
we can reconstruct it using only knowledge of the trajectories in the
upper and lower half-planes separately (this will be Lemma
\ref{lem:sampled-twostrip-meas}).

Suppose $X$ starts in the upper half-plane.  Whilst it remains in the upper
half-plane there is no difficulty.
The difficulty arises when trajectory $\sampled$
hits the time axis.  There is no finitary way to join
it to its continuation in the lower half-plane, since it (almost
surely) crosses level $0$ infinitely many times in any interval of
time after first reaching it.

Instead when the trajectory his $0$ we will continue it with a
Brownian motion \emph{independent} of the web until that Brownian
motion hits $\pm\epsilon$, at which point we will continue the
trajectory from the web again.  Call this ``perturbed'' trajectory
$\resamplede$.

{
\newcommand{\joinernoargs}{\psi}
\newcommand{\joiner}[2]{\joinernoargs_{{#1}{#2}}}
\begin{definition}
  \label{def:resamplede}
  We define $\resamplede$, a ``perturbed'' version of $\sampled$, as
  follows.

  Let $\joiner{s}{t}$ be the increments (between times $s \le t$) of
  some Brownian motion independent of $\webnoargs$ (measurable with
  respect to $\reservoir$, say, where $\reservoir$ is independent of
  $\wholealgebra$).

  \begin{itemize}
  \item Starting from time $s$, follow the path $\web{s}{\cdot}{x}$
    until it hits $0$, at time $S_1$, say.
  \item Then follow $\joiner{S_1}{\cdot}$ until it hits $\pm \epsilon$, at
    time $T_1$, say.
  \item Then follow $\web{T_1}{\cdot}{\pm \epsilon}$ until it hits $0$, at
    time $S_2$, say.
  \item Continue indefinitely in this inductive fashion.
  \end{itemize}
\end{definition}
}

\begin{obs}
  \label{obs:2d-proc}
  A rough description of the coupled pair $(\sampled, \resamplede)$
  follows.

  \FIXME{}{Should this be just the informal description above?}
\end{obs}

The idea is that as $\epsilon$ becomes small $\resamplede$
converges to $\sampled$.
The exact sense in which we will prove convergence of $\resamplede$ to
$\sampled$ is uniform convergence on compacts in probabilty.  We will
write this as $\resamplede \to^\P \sampled$, implicitly taking the
topology of uniform convergence on compacts for the values of $\sampled$
and $\resamplede$ in Wiener space.  Thus we wish to prove the following
lemma (the proof is completed on page
\pageref{proof-of-lem:resamplede-to-sampled}).

\begin{lemma}
  \label{lem:resamplede-to-sampled}
  \statementoflemresampledetosampled
\end{lemma}

Now we will describe how to use this convergence result to show that
$\sampled$ can be reconstructed from the information in the upper and
lower half-planes separately.

$\resamplede$ was defined only in terms of the trajectories in the
half-planes, and the additional independent Brownian
motion.  In formal terms

\begin{obs}
  $\resamplede$ is $\twostripsreservoir$-measurable
\end{obs}

and so the convergence lemma allows us to conclude that we can
recover $\sampled$ itself from the same information.

\begin{cor}
  \label{cor:sampled-twostripsreservoir-meas}
  $\sampled$ is $\twostripsreservoir$-measurable.
\end{cor}
  
But $\sampled$ is independent of $\reservoir$, so in fact we can
remove $\reservoir$ from the tensor product.

\begin{lemma}
  \label{lem:sampled-twostrip-meas}
  $\sampled$ is $\twostrips$-measurable.
\end{lemma}

\begin{proof}
  This is a simple result on tensor products of
  Hilbert spaces (for example (1c1), p5 of
  \cite{tsirelson-completion}).
\end{proof}

\subsection{What does this mean in terms of noises?}

The collection of maps of the form $\sampled$ generates
$\wholealgebra$, yet each is recoverable from the information in
$\twostrips$.  So in fact what we have shown is

\begin{theorem}
  $\wholealgebra = \twostrips$
\end{theorem}

\begin{proof}
  $\twostrips \subseteq \wholealgebra$ by definition, and Lemma
  \ref{lem:sampled-twostrip-meas} gives the other direction.
\end{proof}
}
