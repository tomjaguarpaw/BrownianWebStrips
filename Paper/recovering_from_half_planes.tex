{
\section {Recovering the Brownian web from its horizontal half-planes}

We will show that, given the trajectories of the Brownian web for the
upper and lower half planes separately we can reconstruct the
trajectories for the entire Brownian web.

The difficulty is that when a trajectory $\sampled$, from the upper
half plane, say, hits the time axis there is no finitary way to join
it to its continuation in the lower half plane, since it essentially
oscillates with infinite frequency about the value $0$ immediately
after reaching it.

Instead when the trajectory his $0$ we will continue it with a
Brownian motion \emph{independent} of the web until that Brownian
motion hits $\pm\epsilon$, at which point we will continue the
trajectory from the web.  Call this ``perturbed'' trajectory
$\resamplede$.

The idea is that as $\epsilon$ becomes very small $\resamplede$
converges to $\sampled$ so we can recover $\sampled$ from the
trajectories in the horizontal half planes, and the additional
independent Browian motion.  We prove that the independence of this
additional Brownian motion means that, in some sense, it was not in
fact required, and so we can recover $\sampled$ from the
trajectories in the horizontal half planes alone.

Now we'll make this idea formal.

\begin{notation}
Given a Brownian web $\webnoargs$, fix a starting point $(s,x)$ and write
$\sampled$ for the random process $t \mapsto \web{s}{t}{x}$
(which is a Brownian motion).
\end{notation}

{
\newcommand{\joinernoargs}{\psi}
\newcommand{\joiner}[2]{\joinernoargs_{{#1}{#2}}}
\begin{definition}
  We define $\resamplede$, a ``perturbed'' version of $\sampled$, as
  follows.

  Let $\joiner{s}{t}$ be the increments (between times $s \le t$) of
  some Brownian motion independent of $\webnoargs$ (measurable with
  respect to $\reservoir$, say, where $\reservoir$ is independent of
  $\commafactor{\infty}{-\infty}$).

  \begin{itemize}
  \item Starting from time $s$, follow the path $\web{s}{\cdot}{x}$
    until it hits $0$, at time $S_1$, say.
  \item Then follow $\joiner{S_1}{\cdot}$ until it hits $\pm \epsilon$, at
    time $T_1$, say.
  \item Then follow $\web{T_1}{\cdot}{\pm \epsilon}$ until it hits $0$, at
    time $S_2$, say.
  \item Continue indefinitely in this inductive fashion.
  \end{itemize}
\end{definition}
}

}
