{
\section {Recovering the Brownian web from its horizontal half-planes}

We will show that, given a trajectory $\sampled$ of the Brownian web,
we can reconstruct it using only knowledge of the trajectories in the
upper and lower half-planes separately.

\begin{notation}
Given a Brownian web $\webnoargs$, fix a starting point $(s,x)$ and write
$\sampled$ for the random process $t \mapsto \web{s}{t}{x}$
(which is a Brownian motion).
\end{notation}

The difficulty is that when the trajectory $\sampled$, starting in the upper
half plane, say, hits the time axis there is no finitary way to join
it to its continuation in the lower half plane, since it essentially
oscillates with infinite frequency about the value $0$ immediately
after reaching it.

Instead when the trajectory his $0$ we will continue it with a
Brownian motion \emph{independent} of the web until that Brownian
motion hits $\pm\epsilon$, at which point we will continue the
trajectory from the web.  Call this ``perturbed'' trajectory
$\resamplede$.

{
\newcommand{\joinernoargs}{\psi}
\newcommand{\joiner}[2]{\joinernoargs_{{#1}{#2}}}
\begin{definition}
  We define $\resamplede$, a ``perturbed'' version of $\sampled$, as
  follows.

  Let $\joiner{s}{t}$ be the increments (between times $s \le t$) of
  some Brownian motion independent of $\webnoargs$ (measurable with
  respect to $\reservoir$, say, where $\reservoir$ is independent of
  $\commafactor{\infty}{-\infty}$).

  \begin{itemize}
  \item Starting from time $s$, follow the path $\web{s}{\cdot}{x}$
    until it hits $0$, at time $S_1$, say.
  \item Then follow $\joiner{S_1}{\cdot}$ until it hits $\pm \epsilon$, at
    time $T_1$, say.
  \item Then follow $\web{T_1}{\cdot}{\pm \epsilon}$ until it hits $0$, at
    time $S_2$, say.
  \item Continue indefinitely in this inductive fashion.
  \end{itemize}
\end{definition}
}

\begin{obs}
  \label{obs:2d-proc}
  A rough description of the coupled pair $(\sampled, \resamplede)$
  follows.

  \FIXME{}{Should this be just the informal description above?}
\end{obs}

The idea is that as $\epsilon$ becomes very small $\resamplede$
converges to $\sampled$.
The exact sense in which we will prove convergence of $\resamplede$ to
$\sampled$ is uniform convergence on compacts in probabilty.  We will
write this as $\resamplede \to^\P \sampled$, implicitly taking the
topology of uniform convergence on compacts for values in Wiener space
of $\sampled$ and $\resamplede$.  Thus we wish to prove the following
lemma:

\begin{lemma}
  \label{lem:resamplede-to-sampled}
  $\resamplede \to^\P \sampled$ as $\epsilon \to 0$.
\end{lemma}

$\resamplede$ was defined only in terms of the trajectories in the
horizontal half-planes, and the additional independent Brownian
motion.  In formal terms

\begin{obs}
  $\resamplede$ is $\twostripsreservoir$-measurable
\end{obs}

and so the convergence lemma allows us to conclude that we can
recover $\sampled$ itself from the same information.

\begin{cor}
  \label{cor:sampled-twostripsreservoir-meas}
  $\sampled$ is $\twostripsreservoir$-measurable.
\end{cor}
  
But $\sampled$ is independent of $\reservoir$, so in fact we can
remove $\reservoir$ from the tensor product.

\begin{lemma}
  \label{lem:sampled-twostrip-meas}
  $\sampled$ is $\twostrips$-measurable.
\end{lemma}

\begin{proof}
  This is a simple result on tensor products of
  Hilbert spaces (for example (1c1), p5 of
  \cite{tsirelson-completion}).
\end{proof}

Now $\onestrip$ is defined to be generated by trajectories of the form
$\sampled$, so in fact what we have shown is

\begin{theorem}
  $\onestrip = \twostrips$
\end{theorem}

\begin{proof}
  $\twostrips \subseteq \onestrip$ by definition, and Lemma
  \ref{lem:sampled-twostrip-meas} gives the other direction.
\end{proof}
}
