\newcommand{\toinP}{\overset{\P}\to}
\newcommand{\statementoflemresampledetosampled}{$\resamplede \toinP \sampled$ as $\epsilon \to 0$}
{
\section{Recovering the Brownian web from its horizontal half-planes}

\label{recovering-from-half-planes}

\subsection{\TODO{Basic definitions}{name}}

\FIXME{}{Presumably this needs to be moved earlier}

\newcommand{\restrictupper}{\mathcal{R}_{+}}

\begin{definition}
  For any path $f$ write $\restrictupper(f)$ for $f$ stopped at the
  first time it is outside the upper half-plane.  Then
  $\restrictupper(t \mapsto \web{s}{t}{x})$ is the trajectory of the
  web $\webnoargs$ started at the point $(s,x)$ and stopped at the first
  time it is outside the upper half-plane.

  Define $\upperhp$ to be the $\sigma$-algebra generated by the
  collection of paths $\{\restrictupper(t \mapsto \web{s}{t}{x} : s,x
  \in \R \}$.  Define $\lowerhp$ analogously.
  Write $\wholealgebra$ for the $\sigma$-algebra generated by the
  whole web, i.e.\ by all its trajectories $t \mapsto \web{s}{t}{x}$.
\end{definition}

\begin{observation}
  Note that $\upperhp$ and $\lowerhp$ are independent by Definition
  \ref{def:independent-coalescing-bm}\FIXME{}{Ron notes this is
    not sufficient as stated.}.
\end{observation}

\subsection{\TODO{Recovery argument}{Name}}

\FIXME{}{What am I going to do about this notation?}

\begin{notation}
Given a Brownian web $\webnoargs$, fix a starting point $(s,x)$ and write
$\sampled$ for the random process $t \mapsto \web{s}{t}{x}$
(which is a Brownian motion).
\end{notation}

\begin{theorem}
  $\wholealgebra = \twostrips$
\end{theorem}

We have immediately that $\twostrips \subseteq \wholealgebra$, just by
following the definitions, so we only need the reverse inclusion
$\wholealgebra \subseteq \twostrips$.  Now $\wholealgebra$ is
generated by processes of the form $\sampled$, so our theorem reduces
to the following lemma.

\begin{lemma}
  \label{lem:sampled-twostrip-meas}
  $\sampled$ is $\twostrips$-measurable.
\end{lemma}

\FIXME{This is not immediate.}{Explain why this is difficult}

{
\newcommand{\joinernoargs}{\psi}
\newcommand{\joiner}[2]{\joinernoargs_{{#1}{#2}}}
\begin{definition}
  \label{def:resamplede}
  We define $\resamplede$, a ``perturbed'' version of $\sampled$, as
  follows.

  Let $\joiner{s}{t}$ be the increments (between times $s \le t$) of
  some Brownian motion independent of $\webnoargs$ (measurable with
  respect to $\reservoir$, say, where $\reservoir$ is independent of
  $\wholealgebra$).

  \begin{itemize}
  \item Starting from time $s$, follow the path $\web{s}{\cdot}{x}$
    until it hits $0$, at time $S_1$, say.
  \item Then follow $\joiner{S_1}{\cdot}$ until it hits $\pm \epsilon$, at
    time $T_1$, say.
  \item Then follow $\web{T_1}{\cdot}{\pm \epsilon}$ until it hits $0$, at
    time $S_2$, say.
  \item Continue indefinitely in this inductive fashion.
  \end{itemize}
\end{definition}
}

\begin{lemma}
  \label{lem:resamplede-to-sampled}
  \statementoflemresampledetosampled
\end{lemma}

\[\text{
  $\resamplede$ is $\twostripsreservoir$-measurable
}\]

\[\text{
  \label{cor:sampled-twostripsreservoir-meas}
  $\sampled$ is $\twostripsreservoir$-measurable.
}\]  

\begin{proof}
  This is a simple result on tensor products of
  Hilbert spaces (for example (1c1), p5 of
  \cite{tsirelson-completion}).
\end{proof}

\subsection{What does this mean in terms of noises?}

}
