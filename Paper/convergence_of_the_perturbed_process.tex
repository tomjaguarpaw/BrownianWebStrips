{
\section{Convergence of the perturbed process}
\label{sec:proof-of-lem:resamplede-to-sampled}

\newcommand{\statenoweb}{S^{2D}_{\indepbm}}
\newcommand{\statewebapart}{S^{2D}_{\webnoargs}}
\newcommand{\statewebtogether}{S^{1D}_{\webnoargs}}
\newcommand{\twodim}{Y}

\newcommand{\twodime}{\twodim(\epsilon)}

In this section we prove that
\statementoflemresampledetosampled{}.  This statement depends only on the
joint distribution of $\sampled$ and $\resamplede$.
We therefore define $\twodime = \twodim=(\resamplede, \sampled)$ (generally suppressing
the $\epsilon$ dependence in the notation). Let us describe the distribution of
$\twodim$
 as a two-dimensional random process.

We classify the behavior of the process into three states according to
the behavior of $\resamplede$ with respect to $\sampled$.
\begin{itemize}
\item If $\resamplede$ is in state $\statewebO$
and is coalesced with $\sampled$ we say $\twodim$ is in
state $\statewebtogether$.
\item If $\resamplede$ is in state $\statewebO$
and \emph{is not} coalesced with $\sampled$ we say $\twodim$ is in state $\statewebapart$.
\item If $\resamplede$ is in state $\statenowebO$
we say $\twodim$ is in state $\statenoweb$.
\end{itemize}
$\twodim$ starts
in $\statewebtogether$.  From $\statewebtogether$, $\twodim$
can only transition to $\statenoweb$.
This transition occurs when $\twodim$ hits the origin, as the coalesced
$\sampled$ and $\resamplede$ will continue together until they leave their
current half-plane.
From $\statenoweb$, $\twodim$ can only transition to
$\statewebapart$.  This transition occurs when $\resamplede$
leaves the $(-\epsilon,\epsilon)$
interval (i.e.\ $\twodim$ hits either of the $x=\pm\epsilon$ lines).
From
$\statewebapart$, $\twodim$ can either transition to $\statewebtogether$
if $\sampled$ and $\resamplede$ coalesce (i.e.\ $\twodim$ hits the line $x=y$)
or transition to $\statenoweb$ if $\resamplede$ hits $0$ (i.e.\ $\twodim$ hits
the $x=0$ line).
States and possible transitions of $\twodim$ are summarized in Figure
\ref{fig:twodimtranstab}.

The form of the labels given to the states is justified by the following.

\begin{observation*}
In $\statewebtogether$, $\twodim$ follows the law of a (time scaled) one-dimensional
Brownian motion on the line $x = y$.
In $\statewebapart$ and $\statenoweb$, $\twodim$ follows the law of a
two-dimensional Brownian motion.
\end{observation*}

Additionally observe that by the scale-invariance of Brownian motion,
the distribution of the sample paths of $\twodime/\epsilon$ is
independent of $\epsilon$ (modulo time scaling).

\begin{figure}
\begin{center}
%  \begin{tabular}{c | l || c | c | c | c | }
\renewcommand{\arraystretch}{0.9}
\begin{tabular}{c|c|c|c|c|c|}
\cline{2-6}
 & State & Illustration & Law & Next & Trans. Cond. \\ \cline{1-6}
\multicolumn{1}{|c|}{\multirow{18}{*}{$\resamplede$}} &
\multicolumn{1}{|c|} {\multirow{6}{*}{$\statewebtogether$}} &  & \multirow{6}{*}{equal} & \multirow{6}{*}{$\statenoweb$} & \multirow{6}{*}{hits $0$}     \\
\multicolumn{1}{|c|} {} & {} & {\includegraphics[scale=0.33]{r1d.eps}} & {} & {} &     \\ \cline{2-6}
\multicolumn{1}{|c|} {} &  \multirow{6}{*}{$\statewebapart$} &  & \multirow{6}{*}{indep.} & \multirow{3}{*}{$\statenoweb$} & \multirow{3}{*}{hits   $0$}\\
\multicolumn{1}{|c|} {} & {} & {\includegraphics[scale=0.33]{r2dc.eps}} & {} & \multirow{-3}{*}{$\statewebtogether$} &   \multirow{-3}{*}{hits  $\sampled$}  \\ \cline{2-6}
\multicolumn{1}{|c|} {} & {\multirow{6}{*}{$\statenoweb$}} & {}& \multirow{6}{*}{indep.} & \multirow{6}{*}{$\statewebapart$} & \multirow{6}{*}{hits $\pm\epsilon$}     \\
\multicolumn{1}{|c|} {} & {} & {\includegraphics[scale=0.33]{r2dnc.eps}} & {} & {} &      \\ \hline\hline %\cline{1-6}
\multicolumn{1}{|c|}{\multirow{20}{*}{$\twodim=(x,y)$}} &
\multicolumn{1}{|c|} {\multirow{6}{*}{$\statewebtogether$}} &  & \multirow{6}{*}{equal} & \multirow{6}{*}{$\statenoweb$} & \multirow{6}{*}{$x=y=0$}     \\
\multicolumn{1}{|c|} {\multirow{10}{*}{$x=\resamplede$}} & {} & {\includegraphics[scale=0.33]{s1d.eps}} & {} & {} &     \\ \cline{2-6}
\multicolumn{1}{|c|} {\multirow{10}{*}{$y=\sampled$}} &  \multirow{6}{*}{$\statewebapart$} &  & \multirow{6}{*}{indep.} & \multirow{3}{*}{$\statenoweb$} & \multirow{3}{*}{$x=0$}\\
\multicolumn{1}{|c|} {} & {} & {\includegraphics[scale=0.33]{s2dc.eps}} & {} & \multirow{-3}{*}{$\statewebtogether$} &   \multirow{-3}{*}{$x=y$}  \\ \cline{2-6}
\multicolumn{1}{|c|} {} & {\multirow{6}{*}{$\statenoweb$}} & {}& \multirow{6}{*}{indep.} & \multirow{6}{*}{$\statewebapart$} & \multirow{6}{*}{$x=\pm\epsilon$}     \\
\multicolumn{1}{|c|} {} & {} & {\includegraphics[scale=0.33]{s2dnc.eps}} & {} & {} &    \\
     \hline
  \end{tabular}
\end{center}
\caption{Illustrated states and transitions of $\twodim$}
\label{fig:twodimtranstab}
\end{figure}

\newcommand{\boundarylines}{A_\delta}

Define $\boundarylines=\{(x,y) \ :\  |x-y|=\delta \}$.
To prove Lemma \ref{lem:resamplede-to-sampled} we use the following
property of $\twodim$:

\newcommand{\farpoint}{(P,P)}
\newcommand{\probhitboundaryis}[1]{For given $P > 0$, $\delta > 0$ the probability that $\twodim$
  hits $\boundarylines$ before it hits $\farpoint$ is #1}

\begin{lemma}\label{lem:prob-hit-boundary-o1}
  \probhitboundaryis{$o(1)$ as $\epsilon \to 0$}.
\end{lemma}

We delay the proof of Lemma \ref{lem:prob-hit-boundary-o1} to Section
\ref{subsec:excursions-of-twodim}.

\begin{proof}[Proof of Lemma \ref{lem:resamplede-to-sampled}]

Write $s$ for the time at which $\twodim$ starts.
Lemma \ref{lem:resamplede-to-sampled} is equivalent to: for all $\delta > 0$, $u > s$
\[
\P(\twodim_t \in \{(x,y) \ :\  |x-y|<\delta \} \text{ for all } t \in [s,u])\to1 \text{ as } \epsilon\to 0.
\]
The above statement can be rephrased as

\vspace{12pt}
For all $\delta > 0$, $u > s$,
the probability that before time $u$, $\twodim$ has hit $\boundarylines$
is $o(1)$ as $\epsilon \to 0$.
\DOTHIS{}{Make this display better}
\vspace{12pt}

  We prove this as follows.  For any $\eta > 0$,
  choose $P$ so that the probability that standard Brownian motion
  travels from $0$ to $P$ in time less than $u-s$ is less than
  $\eta$.
  Apply Lemma \ref{lem:prob-hit-boundary-o1} to
  choose $\epsilon_0$ such that, for all $\epsilon < \epsilon_0$, the probability that $\twodime$ hits
  $\boundarylines$ before $\farpoint$ is less than $\eta$.
  Then the probability that $\twodime$ hits $\boundarylines$ before $\farpoint$
  or takes less time than $u-s$ to reach $\farpoint$ is
  less than $2\eta$.
  Thus the probability that $\twodime$ hits $\boundarylines$ before
  time $u$ is less than $2\eta$.
\end{proof}

\subsection{Excursions of $\twodim$}
\label{subsec:excursions-of-twodim}

In this section we prove the following, which is slightly stronger than
Lemma \ref{lem:prob-hit-boundary-o1}.

\newcommand{\loger}{\log 1/\epsilon}

\begin{lemma}\label{lem:prob-hit-boundary-o1loge}
  \probhitboundaryis{$O(\frac{1}{\loger})$}.
\end{lemma}

\newcommand{\origin}{(0,0)}

\newcommand{\excursionstart}{T}
  We begin by introducing the notion of an excursion of $\twodim$.
  Almost surely, the times at which $\twodim = \origin$ (which are stopping
  times) form an infinite discrete collection $\excursionstart_0 <
  \excursionstart_1 < \cdots$. We say ``the probability that an excursion
  hits a set $U$ is $p$'' if $\P(\twodim_t\in U \text{
  for some } t\in [\excursionstart_0, \excursionstart_{1}]) = p$.
  \TODO{}{Provide a justification for the well-definedness of these
    stopping times, perhaps in terms of the states?}
  Observe that by equidistribution this probability is the same when
  $t$ ranges over $[\excursionstart_i, \excursionstart_{i+1}]$, and
  note that the hitting events in question are independent.

\newcommand{\probexcursionhits}[1]{\P\left(\text{an excursion hits } #1\right)}

Our approach to proving Lemma \ref{lem:prob-hit-boundary-o1loge} is to
demonstrate that
\[
\probexcursionhits{\farpoint} \gg
\probexcursionhits{\boundarylines} \text { as } \epsilon \to 0.
\]
This is
realized through the next pair of lemmas whose proofs we delay until Section
\ref{proofs-of-the-excursion-lemmas}.

\newcommand{\Omegaeloge}{\Omega(\epsilon\loger)}

\begin{lemma}
  \label{lem:Phitboundaryline}
  For given $\delta > 0$, $\probexcursionhits{\boundarylines}$ is $O(\epsilon)$.
\end{lemma}

\begin{lemma}
  \label{lem:Pabsorbedandtravelsfar}
  For given $P > 0$, $\probexcursionhits{\farpoint}$ is $\Omegaeloge$.
\end{lemma}

\newcommand{\Oe}{O(\epsilon)}

\begin{proof}[Proof of Lemma \ref{lem:prob-hit-boundary-o1loge}]
  $\twodim$ consists of a sequence of excursions, each of which satisfies
  exactly one of the following conditions
  \begin{itemize}
  \item the excursion hits $\boundarylines$ (with probability
    $O(\epsilon$)),
  \item the excursion does not hit $\boundarylines$ but does hit
    $\farpoint$ (with probability $\Omegaeloge-\Oe$, which is itself
    $\Omegaeloge$),
  \item the excursion does not hit $\boundarylines$ or $\farpoint$.
  \end{itemize}
  The excursions are independent, so the probability that $\twodim$
  hits $\boundarylines$ before $\farpoint$ is
  \[
  \frac{\Oe}{\Omegaeloge + \Oe} = O\left(\frac{1}{\loger}\right).
  \]
\end{proof}

\subsection{Proofs of the excursion lemmas}
\label{proofs-of-the-excursion-lemmas}
\newcommand{\tdh}{\rotproc^1}
\newcommand{\tdv}{\rotproc^2}
\newcommand{\rotproc}{Z}

\DOTHIS{}{Perhaps should introduce more consistency between $\rotproc$
  and $\twodim$ in this proof}
{
\newcommand{\x}{\resamplede}
\newcommand{\y}{\sampled}
In this section we give the proof of Lemmas \ref{lem:Phitboundaryline} and
\ref{lem:Pabsorbedandtravelsfar}. For convenience we rotate (and scale)
$\twodim=(\x,\y)$, defining
\[\rotproc(\epsilon) = \rotproc=(\tdh,\tdv)=\frac{1}{2}(\x+\y,\x-\y).\]

Observe that when $\twodim$ is in $\statewebtogether$, $\rotproc$ follows
the law of a standard one-dimensional Brownian motion on the $x$-axis.
Like $\twodim$, $\rotproc$ has the following ``scale invariance''
property: the distribution of sample paths of $\rotproc(\epsilon) /
\epsilon$ is independent of $\epsilon$ (modulo time scaling).

\begin{proof}[Proof of Lemma \ref{lem:Phitboundaryline}]
Consider the process $\twodim$ between times $\excursionstart_0$ and
$\excursionstart_1$. Our goal is to show that with probability at least
$1-O(\epsilon)$, $\twodim$ arrives in $\statewebtogether$, before hitting
$\boundarylines$. That is because once it arrives at $\statewebtogether$ it
can never hit $\boundarylines$ before hitting $\origin$. Next, we observe the
following two auxiliary claims:

\begin{claim}\label{cl:tdv-together-estimate}
  Whenever $\tdv=0$ the probability that subsequently $\twodim$
  arrives at $\statewebtogether$ before $\tdv$ hits $\pm\epsilon/2$ is at
  least a constant (independent of $\epsilon$).
\end{claim}

\begin{claim}\label{cl:tdv-hitting-back-0-estimate}
  Whenever $\tdv=\pm\epsilon/2$ then there is probability equal to
  $\epsilon/\delta$ of $\tdv$ hitting $\pm\delta/2$ before it hits $0$.
\end{claim}

Claim \ref{cl:tdv-together-estimate} follows from scale invariance, while
Claim \ref{cl:tdv-hitting-back-0-estimate} is a standard martingale result on
Brownian motion (observing that on the relevant time interval $\tdv$ is a
standard Brownian motion).

The reduction of Lemma \ref{lem:Phitboundaryline} to those two claims is
similar to the proof of Lemma \ref{lem:prob-hit-boundary-o1loge}. Claims
\ref{cl:tdv-together-estimate} and \ref{cl:tdv-hitting-back-0-estimate} imply
that between two consecutive times when $\tdv=0$ which are separated
by times at which $\tdv = \pm\epsilon/2$,
\[
\frac{\P(\twodim\text{ hits }\boundarylines)}{\P(\twodim\text{arrives at }\statewebtogether)}
\le \frac{(1-C)(\epsilon/\delta)}{C} =O(\epsilon)
\]
\DOTHIS{}{these probabilities are a bit confusing}
where $C$ is the constant of Claim \ref{cl:tdv-together-estimate}. As the
behavior of $\twodim$ is independent on those intervals, we deduce Lemma
\ref{lem:Phitboundaryline}.

\DOTHIS{}{Why did we need independence?}
\end{proof}
}
\begin{proof}[Proof of Lemma \ref{lem:Pabsorbedandtravelsfar}]
\newcommand{\rotfarpoint}{(P,0)}
\newcommand{\segment}{[\epsilon,1] \cross \{0\}}
We bound below the probability that an excursion hits $\farpoint$,
i.e.\ $\rotproc$ hits $\rotfarpoint$ before returning to $\origin$.
We do this by considering the probability that the excursion takes
the following form: $\rotproc$ travels from $\origin$ to the line
segment $Q = [0, \epsilon] \cross \{\epsilon\}$, then hits the horizontal
axis for the first time in $\segment$, then travels to $\rotfarpoint$,
before returning to $\origin$.

After a stopping time at which $\twodim = \rotproc = \origin$ there is a positive probability $K$
that $\rotproc$ hits $Q$ before returning to $\origin$.
By scale invariance $K$ is independent of $\epsilon$.

Consider $\rotproc$ after hitting some point in $Q$.  We now
bound the hitting density of this process on the horizontal
axis.  Regardless of the point in $Q$, this density for points on
$\segment$ is at least
\[
\frac{1}{\pi\epsilon} \frac{1}{1 + (x/\epsilon)^2} dx.
\]
This follows directly from the classical result that the hitting density
on a line of the two-dimensional Brownian motion is a Cauchy distribution
(see for example Theorem 2.37 of \cite{mortens-peres}).

On hitting a point $(x,0)$ for $x \in [\epsilon, 1]$ the process
transitions from state $\statewebapart$ to state $\statewebtogether$.
When in state $\statewebtogether$, $\rotproc$ behaves as a
one-dimensional Brownian motion on the horizontal axis until it hits
$\origin$.
By the same martingale argument which justifies
Claim \ref{cl:tdv-hitting-back-0-estimate}, the
probability of subsequently hitting $\rotfarpoint$ before $\origin$ is $x/P$.
Integrating this against the hitting density we get that the probability that
$\rotproc$ started from some point in $Q$ hits the horizontal axis in $\segment$
and then travels to $\rotfarpoint$ before returning
to $\origin$ is at least
\[
\frac{1}{\pi P} \int_{\epsilon}^{1} \frac{x/\epsilon}{1 + (x/\epsilon)^2}
\, dx
=
\frac{\epsilon}{2\pi P} \log\left(\frac{1 + (1/\epsilon)^2}{2}\right)
,\text{ which is }
\Omegaeloge.
\]
\TODO{}{This now looks slightly odd}
\end{proof}
}
